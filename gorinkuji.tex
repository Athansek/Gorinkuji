Gorinkujimyôhishaku


Introduction fait

A mon humble avis, le Mandala des cinq roues (Gorinmandara)  et celui du Shingon d'Amida Nyôrai à 9 syllabes sont respectivement l'expérience intérieure (Naishô)  du Grand Souverain (Tei ô/Mikado)  Dainichi Nyôrai et la plus profonde intention de l'Honoré des Mondes (Seson) Amida. Ils sont l'Accès à l'atteinte de l'Eveil sans supérieur en cette vie et l'unique Véritable moyen de l'obtention de la renaissance dans la Terre-pure (Junshi ôjô) .
Pourquoi? Ce qui entraperçoivent ce Mandala ou entendent parler de ce Shingon assistent à l'Enseignement du Bouddha (Kenbutsumonpô) dans cette vie (Oshishô) . Et ceux qui ne serait-ce qu'un instant, méditent (Ikan) à ce sujet seront en fin de compte coupés de la souffrance et obtiendrons l'Extase (Riku dokukaku) dans cette existence même. Ne nous attardons pas alors sur ceux qui possèdent les pures racines de la Foi qui pratiquent avec ardeur. Eux, atteindrons le stade de l'Illumination (Kakui)  de Dainichi Nyôrai aussi simplement que de tourner la paume de la main (Tenohira/tanagokoro) et peuvent prétendre à la renaissance dans la Terre-pure d’Amida le Bien-allé (Zenzei) en appelant (Shomyô) son nom. Si tels sont les mérites (Kudoku)  de la récitation de son nom, il n'est pas possible que les mérites de la méditation sur la Vérité soient inexistants.
Dans l'Enseignement explicite (Kengyô), Amida Nyôrai existe distinctement de Shaka Nyôrai, mais dans la corbeille mystérique (Mitsuzô), Dainichi est identique au Bouddha qui prêche (Kyôshu) dans le Gokuraku: Amida.
Souvenez vous alors, les Terre-pures des dix directions ne sont que qu'un seul même Monde de Bouddha (Kedo). Les divers Ainsi-allés sont tous Dainichi: « Dainichi » et « Amida » sont simplement des noms différents pour un même corps, et « Gokuraku » et « Mitsugon » des dénomination différentes pour un seul et même endroit.
C'est par la sagesse lui venant de sa Sapience-de -merveilleuse-perspicacité (Myôkanzachi), et par la puissance surnaturelle de son Adjuvance(Kaji) que les caractéristiques de Amida apparaissent sur le corps de Dainichi. Tout ce qui en ont atteint un haut niveau de perspicacité savent que tous les Bouddhas, tous les Boddhisattva, quelle que soit l'avancée de leur carrière, les Dragons-célestes (Tenryu) , les Gaki et les huit catégories d'êtres surnaturels (Hachibushû), il n'y en a aucun qui ne soit pas le corps même de Dainichi Nyôrai.
L'Enseignement sur les 5 Roues révèle le Bouddha-en-Corps-de-Loi-de-Nature-Propre (Jisshô Hôsshin)  et le l'Enseignement d'Amida symbolise le Bouddha-en-Corps-de-Loi-de-fruition (Juyûshin). Si ces deux Bouddhas, Dainichi et Amida, sont les mêmes,comment alors distinguer entre leur Sagesse et leur Sainteté (Kensei)?
Les Terre-pures de Gokuraku et Tosotsu sont les demeures du même Bouddha, le Mitsugon Jôdo est lui aussi un pétale du Monde-Matrice-Lotus (Rengezôsekkai) qui est  l'Esprit Indivis (de Dainichi).
Dés lors, on peut déplorer les arguties des Maîtres antiques (Koken) quand à la difficulté, ou la facilité,avec laquelle on peut renaître dans le Paradis de L'Occident, mais qu'il est plaisant que moi, un moine ignorant, j'ai pu y renaître (Toku ôjô)! 
C’est pour cela que j'ai composé ce traité très-profond qui pourrait tenir en une seule phrase: Les difficultés pour naître dans la Terre-pure ne sont causées que par la présence de l’Appropriation (Ushû).
Nous avons établis dix parties qui permettent d’expliquer ces Shingon rapidement:
Déterminer quel est la doctrine (Chakuhô)  qui enseigne que les Voies provisoires et absolues (Gonjitsu) mènent au même but (Dôshu);
 Entrer correctement (Shonyu) dans le Shingon profond (Himitsu Shingon);
 L'obtention (Shôhaku) de mérites (Kudoku) incomparables (Mubi);
 La porte des pratiques (Gyômon)  secrètes (Mitsu) que l'on accompli (Shosa)par soi-même (Jijô);
 Une seule pratique (Ichigyô) en accompli plusieurs;
L'Enseignement sur l'obtention de la plus haute des neuf naissances dans la Terre-pure;
Ôter les obstacles démoniaques à la compréhension éveillée;
Divers Enseignements sur l'atteinte de la bouddhéité dés ce corps;
L'Enseignement sur les capacités de ceux à qui l'on doit enseigner;
Levée des doutes par des questions et des réponses.
Déterminer quel est la doctrine qui enseigne que les voies provisoires et absolues mènent au même but fait
Celui qui désire entrer dans ce Grand Véhicule, suprême, très secret et unificateur, doit pour ce faire: développer le Coeur de Prajna (Hannya shinyashin) et pratiquer (Tôshu) le Triple-mystère (Sanmitsu).
Si un homme ou une femme de bonne famille franchit le seuil de cet enseignement ne serait-ce qu'un moment, les généraux des quatre vingt mille soldats de la souillure (Mei)  mèneront leurs troupes à la soumission. Les cent soixante brigands de l'illusion s'enfuiront avec leurs complices et les quadruples ou octuples transgressions (Shijûhachijû)  seront emportées par le vent. Enfin,les mers agitées des trois ou cinq obstacles (Sanshôgoshô) seront asséchées.
Grâce à ces pratiques, on est libéré (Datsu) en un instant (Setsuna) des entraves karmiques (Gobakû)  qui entraînent dans l'Océan-des-naissances-et-des-morts (Rutenshôji) et les entraves produites par cet océan de souffrances ininterrompues (Kukai) sont détruites instantanément (Suyu). Ensuite les vents illusoires (Môfu) des cinq modes d'existence composite (Gobui) cessent d'eux mêmes, sans que le pratiquant ne les méprise. Les Trois Familles (Sanbû) sont trouvées sans les chercher. Tout cela n'est il pas merveilleux?
Question:
Quels sont les sutras et sastra (Kyôron)  appuient cette manière de penser (Shishin)?
Réponse:
« La distinction dans la Voie bouddhique entre le Vrai et le conditionné (Gonjitsu) ,
L'abandon de l'inférieur pour le supérieur caractérisent la Vérité absolue (Shôgi).
La distinction du profond et du bas-fond (senshin) dans l'océan des phénomènes individués ;
La non-naissance (muki) d'appropriations illusionnées (môjû) sont appelés Hannya.
Le Corps de Loi en félicité qu'est Dainchi Nyôrai, les Bouddha altruistes ;
Kongosatta le vénéré aux merveilleuses vertus(myôtoku), les Bouddha d'Alors, Ryûmyo Bosatsu ;
Toutes ces incarnations du Bouddhas ont expliqués dans leurs écrits comment s'opère cette distinction.
En effet ce sont les Écritures exposées(Sekkyô) par les Bouddhas et leurs Traces qui expliquent le principe de discrimination.
Qui veut atteindre rapidement la Bouddhéité (Tongo) doit pratiquer(Shûgyô) cette attitude spirituelle (Shishin).»
Question: Combien y a t’il de voies bouddhistes? Combien sont provisoires et combien sont absolues? Lesquelles sont étroites et lesquelles sont profondes ?
Réponse en vers:
“L'homme au désirs de bouc ressent la honte quand il prend conscience du mal qu'il commet, c'est le point de départ.
L’enfant ignorant qui comprend les règles de l'abstinence commence  à cultiver les vertus.
Le nourrisson (Yôdômuishin) sans craintes est l’étape des êtres divins (Tenjô) ;
Seulement les Go-on et pas de Soi (Muga), c’est le véhicule des auditeurs ;
Le Bouddha-pour-soi déracine les graines du karma (Gôinshû) ;
Le Mahayana consacré aux autres est la maison du Yuishiki ;
Le stade de la non-production(Fushô) de l'esprit est le Sanron Shû;
Le palais du Lotus est le chemin de la non-existence ;
L’absence absolue de nature-propre est l’enseignement du Kegon ;
L’Enseignement du Shingon est transfiguré par les Mystères.
Les neufs premières étapes sont appelés étroites (Sen) et provisoires (Gon) ;
L’Ultime, l’Esprit-uni est appelé profond et absolu ;
Bien que chacun soit appelé une merveilleuse réalisation, il n’est pas le Bouddha véritable.
Comment un miroir (Enkyô) au milieu de l’eau (Suichû) peut il être la Vérité?
Au surplus, bien qu’appelés des enseignements avancés, ils ne sont que des véhicules imparfaits.
Il n’y a pas de vérité dans les bulles (Hô) de la surface du lac.
Les chemins graduels menant du bas-fond au profond sont les étapes successives de l’abandon de l’inférieur pour la saisie du supérieur.
Question : Quelle est la différence entre le Bouddha des neufs premiers types d’enseignements (bouddhistes) et la divinité du dernier stade, la réalisation du Shingon.
Réponse en vers:
”Le Bouddha exotérique est à la fois ignorant (Mumyô) et non-illuminé (Himyô).
Le Bouddha de l’ésotérisme épuise l’illusion (Meimo) et ne laisse aucun résidu (Muyo).
Le bouddhisme exotérique est une explication faite par les deux corps de transformation.
Le Mikkyo est l’explication faite par le Hôsshin.
L’Exotérique crée un enseignement approprié aux pensée d’autrui (Suzuita).
L'Ésotérique est conçu comme l’enseignement en accord avec les pensées de Dainichi Nyôrai.
L’Exotérique est l’enseignement provisoire et approprié à une situation donnée.
L’Esotérique est l’enseignement absolu, fondamental et final.
L’Exotérique nécessite trois kalpa (Sankô) pour atteindre la bouddhéité.
L’Esotérique consomme la Voie Bouddhique (Shô butsudô) en une vie(isshô).
L’Exotérisme indique le nom d’un ou deux Hôsshin (Hôsshimyô).
L’Ésotérisme, en a quatre voire cinq voire une infinité.
L’Exotérisme contient une partie du principe spirituel, 
l’Enseignement Esotériaue décrit les phénomènes, le Principe (Jiri) et leurs caractéristiques secrètes (Mistsusô) .
L’Exotérisme glose sur la suppression des passions, 
l'Ésotérisme enseigne à la fois la restriction des passions et la mise en pratiques d’actes et de pensées positives (Kyôtoku).
Le Bouddhisme exotérique enseigne les choses bienfaisantes, l'Ésotérisme n’en est  autre que le Samadhi.
Le bouddhisme exotérique explique les causes de la bouddhéité (Inbun kasetsu).
Le Shingon enseigne le résultat de la Bouddhéité (Kabun kasetsu).
Le Bouddhisme exotérique est un océan de pratiques causales menant à l'état de Bouddha.
Le Shingon est l’océan des vertus parfaites de l'état de Bouddha (Toku enmankai).
L’Exotérique explique les bases d’une réalité qui est l’Esprit-Un (Isshinshinnyo).
L’Ésotérisme explique l’égalité des trois mystères (Sanmitsu byodoshi).
Le Bouddhisme Exotérique n’explique pas l’Esprit-Un en toute choses.
L’Ésotérisme explique d’innombrables Esprit-uns.
L’Exotérisme n’ explique pas les Ainsités sans nombre.
L’Ésotérisme est le principe du Filet d’Indra (Juju Taimô).
L’Exotérisme fait du principe la racine de toutes formes (Shushiki).
L’Ésotérisme explique que le principe est la forme indestructible.
L'Exotérisme dit que le principe est ineffable (Mugongo).
L’ Ésotérisme explique que l'expression du principe a d'innombrables modalités.
Dans le bouddhisme Exotérique le Hôsshin n’enseigne pas la Loi (Seppô).
Dans le bouddhisme Ésotérique les quatre corps enseignent la Loi.
L’Exotérisme explique que les Quatre grands (voeux des boddhisattva:Shikyojinguyô) dépassent pratiques et promesses.
L’Ésotérisme établit cinq serments en tant que pratiques et promesses.
Dans l’Exotérisme, le Hôsshin est un et sans aréopage (Mukenzoku).
Dans l’Ésotérisme chaque Hôsshin a un aréopage.
Dans l’Exotérisme le Bouddha du Principe et de la Sagesse n’œuvre pas pour le bien des êtres (Rishô).
Dans l’Ésotérisme, ils œuvrent dans les trois temps au Salut des êtres (Doshyô).
L’Exotérique réalise le principe d'une seule réalité absolue (Shinnyo ri).
L’Ésotérisme réalise d’indicibles et innombrables réalités absolues.
L’Exotérisme explique de sérieux obstacle (Jûsho) qui empêchent l’Eveil (Fukenbutsu).
Dans l’Ésotérisme, bien qu’il ait des obstacles, l’Eveil est toujours obtenu.
Dans l’Exotérisme, sans la contemplation profonde on obtient pas le Fruit.
Dans l’Ésotérisme, en incantant qu’un Shingon on devient aussi un Bouddha.
L’Exotérisme est l’enseignement des Boddhisattvas et des Maîtres Humains (Shi setsu).
L’Ésotérisme est l’enseignement des quatre Hôsshin.
Le bouddhisme Exotérique est l’acmé, le déclin de la Vraie Loi, son imitation et sa disparition.
Le Shingon est l’éternel et immuable (Jôjû fuhen) Enseignement.
Le bouddhisme Ésotérique est l’explication de la production par des causes directes et secondaires (Innen shoshô).
Le bouddhisme ésotérique enseigne la Réalité telle qu’en elle-même (Honi jishô).
Le Bouddhisme exotérique explique un sujet avec un grand nombre de noms et de phrases.
L’Ésotérisme incorpore un grand nombre de sens dans le principe d’une syllabe.
Le bouddhisme exotérique est la parole du Hoshin et du Ojin.
Le bouddhisme Ésotérique est celle du Bouddha pour son propre plaisir (Jiju Horaku).
L’Exotérisme explique la théorie d’un principe et maints phénomènes.
Le bouddhisme Ésotérique pose tous les phénomènes comme étant semblables et différents.
Le bouddhisme Exotérique est l’enseignement qui explique a peu prés les caractéristiques d’une syllabe.
Le bouddhisme Ésotérique est le principe, les caractéristiques et le sens d’une syllabe (Jisô jigi).
L’Exotérisme n’explique pas les quatre mandalas; L’Ésotérisme oui.
L’Exotérisme n’explique pas les Cinq qualités (Gosô); l’Ésotérisme oui.
L’Exotérisme n’explique pas les Cinq Sagesse; L’Ésotérisme oui.
Les six éléments éléments de l'exotérisme sont étroits dans leur explication; dans l'Ésotérisme ils sont profonds et étendus.
L’Exotérisme cache les trois mystères; L’Ésotérisme les explique en totalité.
L’Exotérisme cache les Trois Familles; l’Ésotérisme les explique en totalité.
L’Exotérisme n'explique pas les deux mandalas; seul l’Ésotérisme les explique.
L’Exotérisme n’a pas de théorie de l’Eveil par une contemplation (Ikan).
L’Exotérisme n’a pas de théorie l’Eveil par la visualisation des syllabes (Kanji).
L’Exotérisme n’a pas de théorie de l’Eveil par les mudras.
L’Exotérisme révèle une bouddhéité déjà atteinte il y’a cinq cents Eons d’amas de poussière (Gohyaku jinden).
La Glose secrète enseigne une voie de perfection sans commencement (Fushô jôdô).
A cause de toutes ces différences le Mikkyo est l'Enseignement profond.
Entrer correctement dans le Shingon profond
L’Enseignement sur le corps doté de cinq Roues
La pratique correcte du Shingon-Mikkyo est composée de trois membres:
la pratique du Mystère du corps ;
La pratique du Mystère de la parole ;
La pratique du Mystère de l’esprit.
La pratique du Mystère de la parole en elle-même est triple :
La pratique de la récitation des Shingon ;
La pratique de la méditation sur le forme des Syllabes ;
La pratique de la compréhension des Syllabes.
La pratique de la récitation des Shingons doit se faire avec attention car elle ne peut pas souffrir de fautes.La pratique de la méditation sur la forme des Syllabes se base sur l'aspect extérieur et la forme des graphèmes. Par exemple on visualise le Bija “ON” sur le bout du nez et on atteint l’Éveil lors de la veillée la plus tardive.
Sur la compréhension des Syllabes
Réaliser le sens des  syllabes veut dire comprendre complètement la Vérité sous-jacente à chaque syllabe sous deux rapports :
L'explication sommaire du sens de chaque syllabe qui est l’Enseignement du Hôsshin sur les 5 Roues et les 5 Sapiences ;
L'explication qui embrasse la totalité du Hôkkai Hôshin qui est l’enseignement du Hôshin sur les 9 syllabes et les 9 possibilités de naissance dans la Terre-pure d'Amida.
l'Enseignement sur le Hôsshin
Dainichi est le Hôsshin fondamental. Il entre dans le Samâdhi « L’Essence unique, pouvoir rapide de tous les Ainsi-venus » pour expliquer le Samâdhi de « L’Essence véritable du Hôkkai ». Alors il s'exclame :
« Je réalise le Sans-naissance originel qui transcende tout discours. Tous les péchés se dissolvent. Je ne suis plus soumis aux causes directes et secondaires. Je réalise que la vacuité est comme l’Espace.».
Dainichi Nyôrai est dans le Samâdhi « Jeu adamantin qui détruit les quatre démons »
Il entonne les vers sur la destruction des quatre démons  qui sauve des renaissances dans le Rokudô et fait naître la Sapience-omnisciente :
« NOMAKU SANMANDA BODANAN AK BI RA UN KYAM».
Ces cinq syllabes sont les mots du  Shingon qui détruit les quatre démons.
Explication du Shingon des cinq Roues
Explication de la syllabe « A »
 La première partie signifie prendre refuge dans les trois joyaux. L’«A» signifie  «La Pratique »  et « Sans-naissance fondamental».  Les deux points indiquent la purification, la destruction des quatre démons et de la Souffrance.
A la manière de la Terre, le « A » fait émerger  toutes choses, la Terre du signe « A » fait émerger la myriade des pratiques des six Paramita. « Terre » signifie « Solidité ». La Bodaishin est adamantine, non-régressive et se cristallisera nécessairement en une myriade de résultats propices .
Si le pratiquant lance sa fleur elle atterrit sur la base-mentale (qui est la syllabe « A », la Bodaishin innée) et il plante la graine de l’Éveil. Si il ne subit pas de grande infortune, telles qu'une grave maladie, il atteindra rapidement l’Éveil sans supérieur.
Il prendra prendra alors le nom de « Ichijin Rin Ô ». Ne vilipendez pas ce corps qui est le véhicule vers la Bouddhéité
Il vaut mieux médire de l'Enseignement du Shingon et le remettre en cause, que pratiquer les trois véhicules inférieurs. Alors que dire de celui qui peut lancer la fleur  et pratique avec Foi et Discipline !
 Voici ce que sont la  Croissance et la Maturation de l'  «A».
Explication de la syllabe « BI »
Le Syllabes « BI» exprime la notion de « Lien». Sa représentation avec la syllabe « I » est le Samâdhi sans obstructions qui est la libération inconcevable.
«BA» est l’Eau qui nettoie la fange des souillures,l’esprit et le corps qui pratiquent la Paramita de l’Énergie qui éloigne la distraction par la myriade des Exercices. La syllabe «BA» est l’Eau et le fait de ne pas être dissipé, c’est le parfait Océan des vertus de Dainichi.
Explication de la syllabe « RA »
«RA» est la purification des organes des six sens, la conumation du bois impur de l’action, la purification des trasgressions et des obstructions des organes sensoriels, l'obtention du Fruit.
Explication de la syllabe « UN »
«UN» se décompose en «HA», «Û» et «MA» comme expliqué dans le « Unjigi». Ce sont aussi les trois Accés à la Libération. Le Vent peut souffler les impuretés légères et lourdes. Le Vent du «HA» balaye les quatre vingt-mille impuretés et actualise le principe des quatre Nirvanas. Quand ce vent des causes directes et indirectes se calme, c’est le repos et l’extase du Grand Nirvana.
Explication de la syllabe «KYAN»
« KYAN » exprime l’élément Espace, Le Hôkkai qui englobe tout. l’Espace indestructible n’entrave aucun existant et permet le dévellopement. L’Espace englobe les Terres Pures et souillées et peut mener à perfection les fruits directs et indirects du sage comme du simple d’esprit.  
Zenmui Sanzô à dit:
« Le Cœur du Kongochôgyô, l’œil du Dainichikyô, le champ de mérite sans égal et la vertu superlative sont dans ce Shingon de cinq syllabes. S’il est reçu les vertus obtenues ne peuvent être mesurées. Plus jamais il n’y aura de calamité ni de maladies. Les pires péchés seront éliminées et maintes vertus obtenues. Et votre corps, né de vos parents, éprouve le grand Eveil. Le réciter une fois, vingt-et une fois ou quarante neuf fois équivaut à répéter cent-millions de fois les douze divisions du Tripitaka. »
C’étaitent les éloges de ce Shingon.

Représentations des cinq Roues
Voici des images des cinq éléments, des cinq Roues, de Hôkkai des six éléments, de la roue des dix mondes, du véritable aspect de l’Esprit et de la Matière de tous les êtres et de la représentation de la Bouddhéité en ce corps:













Les cinq Roues ci-dessus sont dits de la couronne de la tête, du visage, du torse, du ventre et des genoux et sont désignés par rapport au corps du pratiquant.
L'Enseignement sur les dix Esprits
Dans le Kongôkai, la syllabe « BAN » se change en les Cinq Roues. Dans le Taizôkai, la syllabe « AKU » est manifestée par les Cinq Roues. Il est dit que “A BA RA KA KYA” devient le Cinq Roues-sekkai. Le pur esprit de foi du gyôja devient les Syllabes des cinq Roues. Cet esprit de pure Foi est pure Bodaishin. C’est cela connaître son esprit tel qu’il est. Verticalement cela exprime les  dix types d’esprits étroits et profonds. Horizontalement c’est l’infinité du nombre de ces Esprits.
Quand je quittais ma province, le mauvais karma des trois poisons faisait mon Esprit pareil à celui du bouc. J'aurais pû dégénérer dans les 8 difficultés et les 3 voies inférieures.
Je connaissais les mauvais agissements de mon esprit et avais quitté le foyer d’ignorance qu'était la demeure de mes parents. Je ne cherchais plus la gloire et la célébrité et j'en vins à avoir Foi dans l’Existant-en-soi illimité, superbe et infini. C'était le premier niveau de la connaissance de l'Esprit tel qu'il est.
Comme Chou Ch’u qui se tenait à l’écart des trois dangers, ou Ajatanatru qui regrettait les trois mauvaises actions, je compris le principe du jeûne et de la modération. Je pratiquais fréquemment les huit abstinences et redoublais mes vœux pour obtenir le succès.
Nonobstant la pourpre du Shishinden, qui était la masure héritée des vertus de ma précédente vie, je vins à observer la quintuple sphère des perceptions externes agréables et je me rendis compte que tout ceci était désagréable. Ainsi j'accomplis le deuxième niveau de la connaissance de l'Esprit tel qu'il est.
La haute plateforme du premier dhyana n'est aujourd'hui que l'objet d'une fierté passée. J'y ai vécu la joie d'être separé de mon Soi naturel, cette sensation ne s'est jamais arrêtée, c'est ainsi que j'ai vraiment su que mon Esprit est enfantin.
J'ai ensuite compris graduellement que ce monde est une maison de bois en flamme. Je somnolais dans la chambre des Shômon et des Engaku. Je finis par connaître la faible éradication qu’engendrent ces deux véhicules mais aussi que ce n'est que faire l’expérience du principe de la nature foncièrement vide des êtres vivants.
J'ai fait l'expérience de l’Esprit-dédié-aux-autres et de l'arrêt des verbalisations et je me suis rendu compte que les écoles qui vantent ces pratiques postulent différentes natures d'Humains.
La tradition de la Bodaishin-jamais-née n'arrête que les délibérations au sujet de la vacuité. Ainsi j'ai pris conscience des maladies nommées « Existence » et « Vacuité ». Alors je me suis empressé de laisser derrière moi les trois grands éons. Mais cela équivalait à attendre qu'un rocher s'érode jusqu'à devenir poussière ou qu'un tas de graines de moutarde soit complètement dispersé.
L’Eternel Buddha du Myôho Renge Kyô, démontrait son origine il y’ a cinq cent grands éons de cela et celui du Kegon Kyô , ne prêchait pas lui même le Dharma. C’était une compréhension partielle de l’esprit tel qu’il est.
Au contraire, l'Enseignement secret de la méditation en cinq étapes et les cinq Sapiences, l’ornement du Monde-principe et le stade de l’esprit originel de l’Esprit-d’éveil-en-soi sont l’illumination naturelle. C’est aussi connaître son esprit tel qu’il est. La signification profonde sera expliquée plus tard.
L'Enseignement sur la Matière et l'Esprit
La vraie caractéristique de la forme et de l’esprit de tous les êtres est le Corps-de-sapience-de-l’égalité-de-l’origine sans commencement de Dainichi.
 La  Matière est le Skhanda de la forme. Développé il devient les Cinq Roues. L’Esprit, lui, est l’élément conscience qui devient les quatre autres Skhanda. Matière et Esprit sont identiques au Rokudai Hôsshin qui est la Sapience de la Nature du Hokkai.
Les cinq Roues sont appelés « Roues » car ils sont doté de la myriade des vertus. Leurs caractéristiques intrinsèques sont sans limites, c’est pour cela qu’elles sont qualifiées de “grandes”. Parce que les Gobutsu sont illuminés par eux-mêmes et illuminent les autres, on les appelle Nyôrai.
Parce que les Cinq Sapiences font la distinction entre les Objets, on les appelle Sapiences.
Si la Matière est unie à l'Esprit, les Cinq Grands Eléments sont vus comme identiques aux Cinq Sapiences. Si la Matière est unie à l'Esprit, Les Cinq Sapiences sont vues comme identiques aux Cinq Roues.
Parce que « la Forme est la Vacuité », la myriade de dharmas est identique aux Cinq Sapiences.  Parceque « la Vacuité est la Forme », les Cinq Sapiences sont identique à la myriade des dharmas.
La Matière (rupa skhanda) et l'Esprit ne sont pas deux, donc les Cinq Eléments sont identiques aux Cinq Organes, et les Cinq Organes sont identiques aux Cinq Sapiences.
Expositions des correspondances des Cinq Eléments
Exposition des correspondances selon Zenmui Sanzô

Exposition des correspondances selon Fuku Sanzô


Exposition des correspondances transmises à Kakuban


Ci-dessus sont les Cinq Bouddha et les Cinq Sapiences de l'unité de la Vérité et de la Sagesse du Mitsugon Jodô. Identiques au Kongôkai,  ce sont les  Cinq Bouddhas et les  Cinq Sapiences du Taizôkai.
La révolution de la sylabe « BA »


Le tableau ci-dessus est l’Esprit, les Cinq Bouddha du Kongokai et son Grand Véhicule unificateur. Identiques au Taizôkai ils sont néanmoins les cinq Bouddha non-duels et les cinq sapiences du Kongôkai.
Si on connait les cinq Roues de cette syllabe “BA”, on connait celles des autres syllabes. Comme chacune porte en elle les cinq Sapiences, elles sont en fait dotées de Sapiences innombrables. En raison de l'action de la Grande-Sapience-telle-le-miroir-parfait, chacune d'elles sont véritablement des Sapiences d'Evéillé. C’est la bouddhéité-atteinte-par-son-propre-esprit.
Si le gyôja, lors des 4 périodes de la journée, n’est pas interrompu, peu importe qu’il soit réveillé ou endormi, demeure dans la Sapience et la Prajna, et pratique ce Samadhi il réalise Sokushinjobutsu sans difficulté.
Le yoga transmis par Kongôchi Sanzô


Sanzô a dit :
 « En pratiquant cela avec Foi pendant mille jours, soudainement lors d’une nuit de pleine lune durant l’automne, j’ai atteint le Samadhi « Elimination des obstacles ».
Lorsque moi, Kakuban,  j’ai entendu ce secret, j'y prété foi. Pendant de nombreuses années je l’ai pratiqué et atteint le Samadhi du premier étape.
Ô pratiquant fidèle ne doute pas et ne nourrit pas de phantasmes. Si mes mots sont vides, tu le sauras en pratiquant. Ce que je désires pour toi est que tu ne vives pas en vain.
L'exposé détaillé des correspondances liées aux cinq organes
Exposé sur le Foie
L'“A” est Ashuku Nyôrai du Kongôbu qui rêgne sur le foie et la Conscience Visuelle. C’est à dire qu’il est le principe dharmakaya de Dainichi Nyorai, la  pureté-en-soi, le royaume originellement-non-né et la vacuité incompréhensible. C’est la Syllabes de la Roue de la Terre de Grande Compassion, et le Kongobumandara. La Terre est la forme . C’est le skandha de la conscience qui soutient la Terre. Quand les deux entrent en contact, s'influençant mutuellement ils engendrent le Désir et l'Existence. Le Vent et l'Espace sont profanés tout comme les Portails de la Terre et du Feu. L'Eau, l'Espace et les semences des Consciences descendant dans la Matrice engendrent les Cinq organes.
Parceque des cinq skandha, c'est celui de la Conscience qui a provoqué la nouvelle renaissance, il est appellé Terre et c'est Rupa Skandha.
Le Foie gouverne “l’Esprit subtil”. Le Ki de l’âme est l’esprit du Bois à l’orient. Sa couleur est le bleu de l'Espace. Cette couleur bleue est tributaire du bois pour advenir et la croissance du bois dépend de l’eau. Le Foie dépend du Souffle bleu et des Reins. Sa forme est celle d’une feuille de lotus, on y trouve une perle de chair et il se trouve à gauche dans la poitrine.
Le foie remonte à la surface du corps et forme les yeux qui gouvernent les Tendons. Quand les tendons s’épuisent ils deviennent des clous.
Yômyô Enju a dit : 
« Le Lotus-Foie a huit pétales bleus qui contient les cinq couleurs principales ».
Exposé sur les Poumons
 “BAN” est le Rengebu d’Amida Nyôrai qui reigne sur le poumon et la Conscience olfactive. C’est la onzième transformation de “BA” ,et “BI” en est la troisième. C’est l’eau de Sagesse de Dainichi Nyorai. C’est la Syllabes de la Grande Roue de l'Eau compatissante d’Amida Nyôrai.
La maîtrise des pouvoirs spirituels est appelé le Hôsshin. Sa conjonction yogique avec son objet relatif est appellée Hoshin. C’est le Hokkebumandara.
Les Poumons gouvernent l’Esprit Grossier et en sont la matérialisation. Son apparence est celle du nez. C’est la direction de l’ouest et le métal. Cet Esprit gouverne l’autumne et sa couleur est le blanc.
On trouve naturellement le souffle entre les Poumons et le nez, c’est çà dire le Vent. C’est le skandha de l’Idée parmis les cinq il maitient le Vent. Le skandha du Vent (fuon) à besoin de la conscience pour apparaître. La Conscience a besoin des deux causes du passé (avidya et samskara) pour produire les 5 effets (go.on). C’est à dire que sur l’ignorance et les formations mentales poussent la conscience, les nom-formes etc. La pensée erronée enfle et produit le samsara. Ce sont les 12 segments de la coproduction conditionnée.
Les Poumons engendrent le conscience mentale. La Conscicence génére les pensées qui à leur tour engendre le samsara.
Quand le Ki blanc et les odeurs fortes entrent dans les Poumons, le foie est endommagé. S'il n’y a pas d’Esprit Grossier dans les Poumons, il y’a l’agitation et la maladie. Si le Coeur endommage les Poumons, la maladie advient.
Tout comme le feu est supérieur au métal, si le Coeur est fort et les Poumons sont faibles, les Poumons sont arrétés par le Coeur.
Si le Ki Blanc surpasse le Ki Rouge les maladies des Poumons sont allégées. Le Ki Blanc est le nom des Poumons.
 Le Lotus-poumon a trois pétale, il est de couleur blanche et de forme semi-circulaire. Ils se trouvent à 4,5 centimètres de part et d’autre de la troisième côte.
Exposé sur le Coeur
« RAM » est le Hôbu de Hôdo Nyôrai qui rêgne sur le Coeur par la bouche. Autrement dit « RAM » est le feu de sagesse de Dainichi Nyorai, la grande compassion de Hôdo Nyôrai, le mandala du Fukutokushin et la semence de l’élément Feu.
Il brûle des souillures de l’Ignorance-sans-commencement et fait germer la graine de la Bodaishin. C’est en somme le Fukutokushin de tous les Bouddha. Le feu de la véritable sagesse consume les causes karmiques de la déshérance et donne la maîtrise des pouvoirs extraordinaires.
Le feu du Coeur rêgne sur l’été et sa couleur est le rouge. La couleur rouge engendre le feu à partir du bois. C’est le skandha de la sensation qui soutient le feu. La sensation dépend de la pensée pour advenir et le Coeur dépend du souffle  rouge et des Poumons pour exister. Le Coeur exteriorisé est la langue qui gouverne le sang. Epuisé le sang se change en lait.  
Il gouverne aussi la perception auditive et produit les passages nasaux, l’arête du nez, le front, les machoires etc. Quand de trop nombreuses saveurs amères pénetrent le Coeur, il enfle et entrave les Poumons. Si l’Esprit ne se trouve pas dans le Coeur, le passé et le futur seront oubliés.
Si les Reins blessent le Coeur il tombe malade. Comme l’Eau l’emporte sur le Feu , si les Poumons sont fort le Coeur est faible. Les Poumons arrêtetent le Coeur.
Si le Souffle Rouge surpasse le Souffle Noir, la maladie du Coeur se résorbe. Le Souffle Rouge est un nom du Coeur. La fleur du Coeur est rouge et de forme triangulaire. On le trouve exactement à 4.5 centimètre à gauche et à droite de la cinquième côte.
Exposé sur les Reins
La syllable « HAM » est la section du Karma de Fukujôju Nyôrai, qui dirige les Reins et l'Estomac. Cette syllabe « HAM » est la 11° tranformation de la syllabe « HA ». On l’interprète comme les autres syllabes et elle porte la signification des 15 voyelles du Shittan, c’est la vie éternelle de Dainichi Nyôrai, la Base Fondamentale de Shaka Nyôrai, la semence du Vent et la Grande Compassion, les 3 Voies de liberation, l'incompréhensibilité des Trois Temps et le Katsuma Mandara du Corps Karmique.
Comme le Vent, la Conceptualisation(des cinq facteurs mentaux omniprésents) supporte l'Eau pareille à un Océan des Cinq Organes et Six Entrailles. Les Poumons et le Vent se rencontrent et la Mer des Renaissances.
Les Cinq Organes sont : le Foie, les Poumons, le Cœur, la Rate et les Reins. Estomac est un des noms des Six Entrailles, il est l’abdomen qui contient la Rate.
L’Eau de l’Océan des Six Entrailles entre dans la cavité de l’Estomac.  Les débordements issus des Cinq Organes et des Six Entrailles sont reçus par l'Estomac.
Les Cinq Saveurs s'écoulent et débordent dans l'Estomac et les Reins reçoivent le trop-plein de celui-ci. Ils se trouvent 3,8 cm de part et d'autre de la colonne vertébrale sous la quatorzième vertèbre,  celui de droite est appelé « Porte de la Vie ».
Les Reins se trouvent dans la partie inferieure de l'Abdomen. S’ils s'épuisent ils se changent en Eau. Elle gouverne la Volonté et devient le Nord et l’Eau qui gouverne l’Hiver qui est de couleur Noire. C'est le skandha de l’Impulsion (samskara).
Des cinq Skandhas, Samskara soutient l'Elément Eau. Samskara est produit par Vedana qui provient de Smajna.
  Les Reins proviennent de la couleur noire, et des Poumons. Ils gouvernent les oreilles. Ils se manifestent dans les os et gouvernent la Moelle. Une fois épuisée, cette dernière devient la cavité des oreilles. Les os affaiblis deviennent les dents.
Quand trop de saveurs salées entrent dans les Reins, ils enflent et  le Cœur s'étiole. S’il n’y a pas de Volonté dans les Reins, un profond regret s’installe. Si la Rate blesse les Reins, la maladie s'installe. Comme la Terre est supérieure à l’Eau, si la Rate est forte, les Reins sont faibles. La Rate arrête les Reins. Si le Ki noir submerge le Ki jaune, alors les maladies des Reins guéries. « Ki Noir » est le nom de l'Eau.
Exposé sur la Rate
La syllabe « KYAM » est le Butusbu de Dainichi Nyôrai de la région supérieure qui conditionne la Rate et la perception de la langue. C’est à dire que « KYAM » est l’Usnisa invisible de Dainichi Nyôrai, les cinq Buchô, la Sapience de l’Espace immense, le Nirvana, l’Ainsité, l’Eveil obtenu par les Bouddha des dix directions et des Trois Temps. C’est aussi le le suprême Mandala sans équivalent.
La Rate gouverne l’Anmara Shiki et se trouve très précisément au milieu. Elle a prise sur l’Eté et sa couleur est jaune. La syllabe « A » est de la vraie couleur Or. La couleur jaune nourrit le Bois dans la Terre et crée le Feu par le Bois. C’est le Skandha de l’Esprit et supporte la Terre. C’est la Conscience Bois qui est bleue et est le Vide. La Rate à besoin de l’Essence Jaune et du Cœur pour exister.
C'est le Vijnana Skhanda qui sert de base à la Terre. D'aucuns disent qu'il est l'organe de Bois (le Foie).
Quand trop de saveurs sucrées y entrent, la Rate enfle et endommage les Reins. Si il n'ya pas d'Esprit Spiritel, la confusion s'installe avec l'illusion. Si le Foie endommage la Rate la maladie fait son lit.
Comme le Bois est supérieur à la Terre si les Reins sont forts la Rate est faible. Le Foie arrête le Coeur.  Si le Ki Jaune surpasse le Ki Bleu les maladies de la Rate son guéries. « Ki jaune » est le nom de la Rate.
Le Lotus-Rate a un pétale jaune à quatre coins.
Conclusion
Les cinq organes sont comme une fleur de Lotus tournée vers le bas. Les cinq organes internes se manifestent par les cinq indriya et prennent une apparence qui est appelée « Forme ». Elle est les cinq éléments et les cinq sens. Son nom est « Pensée » le quatrième des skandha.
Forme et Esprit sont Hôsshin des six éléments, l’Ainsi-allé des Cinq Sapiences, les cinq grands Bodhisattva (Kongo Haramitsu, Kongosatta, Kongohô, Kongohô, et Kongoyô) et cinq grands Vidyarajas (Fudo, Shôsanze, Gundari, Daitoku, Kongoyasha).
Le Soleil, la Lune, les cinq planetes, les douze constellations et 28 loges lunaires deviennent le corps de l'Homme.
L’Elément Terre qui forme iles et montagnes prend naissance dans la syllabe « A ». Toutes les rivières et les courants océaniques viennent de la syllabe « BAN». L’Or et les gemmes précieuses, les constellations, le Soleil et la Lune, les pierres radieuses sont le produite de la syllabe « RAN ». Les cinq céréales, tous les fruits et les fleurs dépendent de la syllabe « KAN » pour arriver à fruition. Les beautés excellentes et parfumées, la croissance des gens et des animaux, la couleur du visage, sa radiance, de bonnes proportions, les grâces, les vertus, la richesse et les honneurs, dépendent de la syllabe « KYAN ».
Louanges du Shingon de cinq syllabes
« A » est l’essence de la pensée, de la vacuité et du Nirvana. En comprenant ceci il n’y a rien à comprendre. En rejetant cela il n’y a pas de rejet. C’est le principe à l’origine de l’Existence et la sapience essentielle du Kanjô.
Ce qui vient d’être dit pour la syllable “A” peut être fait pour les autres. Ces Syllabes sont une expression des parties essentielles du Dainichi kyô et du Kongocho Kyô et sont de profonds et merveilleux champs de mérite et une vaste et vertueuse essence.
Tous les sutra prêchés par les corps de jouissance et les corps de rétribution sont  contenus dans ces seules cinq syllabes. En les récitant seulement une fois les mérites obtenus sont innombrables et mystérieux. De plus tous les bénéfices de l’élimination des calamités, de l’accroissement des gains, de l’assujettissement, de la soumission et de l’attraction sont obtenus.
Le Shingon de cinq syllabe est l’incantation de tous les Bouddhas. Les cinq mudras sont les mudras de tous les Boddhisattva.
Celui qui les pratique doit savoir qu’ils arrêtent pour toujours les calamités et les maladies. Ils sont l’Usnisa orné de joyaux des cinq Bouddha, la base profonde des cinq sapiences, le principe Mère des Ainsi-allés les dix directions œuvrant pour tous les êtres, le Père protecteur des Sages des trois temps. Au surplus ils sont l’Essence intégrale des six éléments et les quatre mandalas, les quatre Corps et les Trois Mystères, le Refuge des quatre sortes de sages, les six sortes d’êtres, la vraie nature des cinq destinées et les quatre modes de naissance. Ils détruisent les quatre Maras et libèrent des six destinées.
Le Yoga du Coeur de Lotus
La syllabe « A » est la Terre adamantine et doit être contemplé en tant que Siège adamantin.
La syllabe « VAM » est l’Eau adamantine et doit être contemplée en tant que Lotus-socle de l’Esprit.
La syllabe « RAM » est le Feu adamantin et l'objet de la contemplation sur le disque solaire.
La syllabe « HAM » est le Vent de la Sapience adamantine et l'objet de la contemplation sur le  disque lunaire.
La syllabe « KHAM » est la samadhi adamantin et la contemplation sur la Grande Vacuité.
Résidant dans la demeure de l’Espace on réalise le Mystère du Corps qui est boire  l’Ambroisie de la non-naissance et la potion de la douce Nature de Bouddha.
Si une syllabe entre les cinq organes les ennuis et la maladies cessent. C'est pour cela que Kôbô Danishi a écrit « Si une seule syllabe entre dans les organes aucune maladie innombrables ne peut advenir ».
Si l'on contemple les disques lunaire ou solaire atteint à la bouddhéité dans cette forme acutuelle non éveillée.  Le cœur a la forme d’un lotus fermé. Il s’agit d’un organe divisé en huit parties par ses veines et ses muscles : les huit pétales du lotus du cœur et les huit parties de la chair.
Contempler le lotus du cœur déploie les pétales du lotus blanc.  Au dessus du réceptacle , contemplez la syllabe « ÂNKU » de la couleur du Vajra. Ceci n'est autre que l'Upaya embrassant totalement l'Ultime, le Dainichi Nyôrai de l'Esprit, la sapience intrinssèque au Hôkkai, le Hôsshin fondamental et éternellement quiescent qui est la nature fondamentale du réceptacle floral du lotus qui surpasse les pétales, il n’est ni royaume de la pensée ni de celui de la parole. Seuls les Bouddhas, et seulement eux peuvent réaliser cela.
Avec cet expédient, toutes les images qui apparaissent sont identiques la Vacuité. L'Esprit placé au centre du Lotus est vide mais embrasse toute la forme. C’est le Hôsshin sans cause qui manifeste les formes.  En d’autre termes c’est l’Océan-Assemblée universel du monde de la grâce des dix directions. Ce n’est pas un endroit hors d’atteinte. En toute place il est en union avec le Hôkkai : il est entièrement la forme du corps d’essence de Dainichi Nyôrai. Comme il est doté de toute les vertus, il est le Bouddha.
Tous les Bouddha sont l’Être Dainichi. Les Deva, Rakshasa, démons et autres esprits sont aussi des caractéristiques du Hôsshin. Aussi, réalisez que ces cinq syllabes sont les incantations de tous les Bouddhas.
Louanges du Yoga et du Yogi
Celui transmet cet Enseignement devrait lui faire des offrandes, comme on le fait pour un Caitya. Il devrait être paré des vertus d’un Arhant. Combien plus devrait’il avoir la Foi et pratiquer ! Il devrait être un lotus blanc parmis ses semblables, une reliques du Hôsshin, il devrait unir les quatre corps de Dainichi Nyôrai, c'est-à-dire être identiques aux cinq sapiences de tous les Bouddhas qui sont pures et innées.
Les neufs consicences de ma nature surgissent des deux formes de karma qui résultent du passé, sont non-duelles et sont de même nature. Les trois Mystères du Mahasattva du Shingon ont ces mêmes caractéristiques et sont comme l’Espace profond. Si le Hôkkai est un palais, le lieu de pratique est le Mistugonjodô. Si il’y a un Honzon composé des sic éléments, alors c’est les êtres vivants. Si la divinité pcerclescipale et le gyôja sont fondamentalement égaux, alors je réalise l’Origine. Je suis l’Ancient Bouddha. Le Monde du principe et de la Sapience sont mon mandala fondamental. Les triple ou quintuple familles sont mon corps.
On peut ignore les cinq éléments, construire le château des trois mondes et transmigrer dans les cinq organes, les cinq éléments et le samsara. Faisant de la syllabe « A » la base de l’ignorance, on souffre sans répit. Si on réalise les cinq éléments, on construit lesaspects des quatre mandalas et réalise les cinq Bouddha, les cinq sapiences et le Nirvana. Cela est l’Origine. Tout le cosmos retourne vers la syllabe « A » et y entre. Les caractéristiques des Enfers et des Cieux, la nature de Bouddha et les ichantika, les souillures et l’Eveil, samsara et nirvana, les vues non-bouddhistes et la juste voie du milieu, la vacuité et l’exitence, le relatif et l’absolu, les deux véhicules et le véhicule unique, peines et joies sont tous le résultat karmique des six éléments. C’est l’interaction des six éléments qui produit les causes et par la sensationdes ces six elements les effets sont produits. Grâce à la vraie sagesse on peu réaliser les six éléments.
Grâce aux Cinq Sapiences, aux Quatre Corps, aux Quatre Mandalas, par la pratique assidue  (Kunshu) du Triple Mystère et en faisant preuve du bon état d’esprit (Zenshin) on témoigne (Nôshô) de l’Inépuisable Transfiguration du Grand Mandala (Mujin Shôgôn Mandala) . Par un attachement irraisonné (Môju) on est induit dans les cinq destinés. Le Samsara, les souillures, les quatre ou huit péchés, les cinq offenses difficiles à redresser et la médisance parfument (Kunjû) l’Esprit qui produit de mauvaises pensées. Grévé par la mélancolie, on endure l’Atroce Grande Rétribution (Daikukan) en chutant au Naraka. Le Soi est aussi bien Illumination qu’l’Inscience (Meigo). En l’absence d’Appropriation (Mushû) on atteint (l’autre rive). Il en va de même pour les Hiérogrammes restants.
Ensuite, sous l’angle du Kongôkai « VAM », « HUM », « TRAH », « HRIH », « AH » purifient les Cinq Organes. Nommément, il s’agit du Foie, du Cœur, de la Rate, des Poumons et des Reins.
Le Foie est bleu et gouverne le Bois. « HUM » en est la Base d'Eveil. « HRIH » peut le contrecarrer. Pourquoi en est-il ainsi? “HUM” est l’Agent Bois, qui ne fait qu’un (Zoku) avec le Foie. “HRIH” est le Métal soit le bija des Poumons. Comme le Métal est supérieur au Bois, les Poumons sont supérieurs au Foie. Ainsi sachez que « HRIH » peut détruire « HUM ». Le Gyôja devrait contempler le sens de la syllabe « HRIH ». Visualiser la coleur blanche, de principe du Foncièrement-non-né. Alors le Métal de l’épée de Sapience. Elle détruit la nature de Bois des trios nuances de bleu/vert : les vues erronées. Elles sont les cinq obstacles, les cent soixante esprits nés de l’ignorance et les vues erronées sur les caractéristiques de la syllabe « HUM ». Tout ceci est épuisé.  Cela nourrit le grand arbre la bodaishin des cinq vajra de sapience originellement non-nés (le sens de l’ « HUM ») et le Seigneur de l’arbre Sala. Cette forme grandit graduellement et devient Ashuku Nyôrai de Adarsa Jnana. Cela dévoile la porte du samadhi adamantin de la Bodaishin.
Le Cœur est rouge et gouverne le feu. Le « TRAH » est son Base d'Eveil, il peut être détruit par l’ « AK ». Pour quoi ? La nature de l’Eau est supérieure à celle du Feu, ainsi les Reins sont supérieurs au Cœur, ainsi le sens de la syllabe « AK » détruit les caractéristiques de la syllabe « TRAH ». Conséquement avec  l’Eau noire des Cinq Sapiences originellement non-nées on douche le feu des trois nuances de rouge : les opinion fausses, c’est  à dire les cinq obstacles, les cent soixante esprits nés de l’ignorance et les vues erronées. Cela produit le rouge Feu adamantin des bénédictions et vertus de cinq sapiences originellement non-nées de la syllabe « TRAH ». Cela enfle et l’embrasement du Feu de sapience devient Hôshô Nyôrai de la samata jnana. En d’autre mots cela révèle la porte du Samadhi adamantin des bénédictions et vertus.
Les poumon sont blancs et gouvernent le métal. Le « HRIH » constitue leur Base d'Eveil et « TRAH » peut les détruire. Pourquoi ? Parce que la nature du Feu est supérieure à celle du Métal et le Cœur est supérieur aux Poumons. La signification de « TRAH » détruit les caractéristiques du « HRIH ». Ainsi avec le Feu rouge de la sapience quintifide originellement non-née on consume le Métal frustre mépris des trois tintes de blanc : les cinq obstacles, les cent soixante esprits nés de l’ignorance et les vues erronées. Il se transmute dans le métal blanc de la Vérité, la sapience quintifide originellement non-née, qui se purifiant devient Amida Nyôrai de la pratyaveka jnana et la porte du samadhi adamantin de la sapience.
Les Reins sont noirs et gouvernent l’Eau. Leur Base d'Eveil est l’ « AK », le « RAM » peut le détruire. Car la Terre est supérieure à l’Eau, et la Rate est supérieure aux Reins. Le sens de la syllabe « RAM » détruit les caractéristiques de la syllabe « AK ». Donc, avec  la Terre indestructible du Vajra quintifide originellement non-né, videz les trois sortes d’Eau noire que sont les les cinq obstacles, les cent soixante esprits nés de l’ignorance et les vues erronées. L’on maitrise alors les huits vertus de la Sapience de vajra quintifide. En débordant petit à petit cela engendre Fukujôjû Nyôrai de la krtyânusthana jnana, la porte du Samadhi adamantin de l’action.
La Rate est jaune et gouverne la Terre. Son Base d'Eveil est le « VAM » qui est soumise au « HUM ». Pourquoi ? La nature du Bois est supérieure à celle de la Tere. Le Foie est supérieure à la Rate. La siginfiaction de la syllabe « HUM » détruit les caractéristiques de la syllave « VAM ». Ainsi avec le fue de la Sapience de vajra quintifide on détruit la Terre erronément saisie des trois uances de jaune, autrement dit les les cinq obstacles, les cent soixante esprits nés de l’ignorance et les vues erronées. Cela produit la Terre originellement éveillée, jaune, pareille a Narayana de la Sapience de Vajra quintifide de la lettre « VAM ». Cela croit et engendre Dainichi Nyôrai de la Sapience intcerclessèque du Hôkkai. Ceci est la Porte du  Samadhi adamantin du Rokudai Hôkkai.
Si lors de la psalmodie, c’est le Kongokai, le corps entre dans le samadhi de Kongo Haramistu et l’on devient cette divinité. C’est l’Essence du Nirmanakaya. Si c’est le Taizôkai, le corps entre dans le samadhi de Monju Bosatsu, la divinité devient un Nirmanakaya. Avec les autres on forme une profonde assemblée, qui est le Hôshin indestructible.
Au moment de la Réalisation, la Vésicule biliaire devient et prend la nom de Gôsanze Myôo. Le Gros intestin au moment de la Réalisation devient et prend le nom et devient Gundari Myôo. La Vessie au moment de la Réalisation devient et prend le nom de Daiitoku Myôo. Le Petit intestin, au  moment de la Réalisation devient et prend le nom de Kongo Yasha. L’Estomac au moment de la Réalisation, devient et prend le nom de Fudo Myôo.  Le Triple réchauffeur, lors de la Réalisation, devient et prend le nom de Fugen Bosatsu.
Saga Tenô demanda « Qu’elle est la preuve de l’atteinte de Sokushin Jobutsu pour le Shingon Shû ? » Avec révérence le Moine (Kûkai) entra dans le Samadhi de la contemplation des cinq organes.. Soudainement apprau sur la tête du moine la couronne précieuse des cinq Bouddha, et son corps de cinq substances irradia une lumière quintuple. Alors l’Empereur se leva de son siège et toute l’assistance s’inclina devant Kûkai. Tous le monastères agitèrent des bannières et l’impératrice fit présent de robes. Ainsi le samadhi des cinq organes était le secret des secrets. Kûkai se leva pas ; un samadhi était expliqué à ce moment. Quoi qu’il en soit cette manifestation était proportionnelle à sa Foi.
Dans les Stances il est dit que quand une syllabe entre dans un organe, toutes les maladies y cessent et on réalise Sokushin Jobutsu. Si un Sage ou un quidam reçoit le Kanjo, forme le Mudra su Stupa avec ses main, récite dans la bouche le Shingon « VAM » en contemplant qu’il dAinichi Nyôrai , sans aucun doute celui-là éliminera l’ignorance comme les cinq péchés séditieux, les quatre péchés majeures, les sept péchés seditions violant les vœux, la calomnies des Sutras du Grand Véhicule, les innombrables péchés des Icchantika etc. Tout sera éliminé.
En l’absence de la plus petite péché, la Bouddhéité dés ce corps est atteinte et l’on est à jamais libéré du samasara. On bénéficie au Êtres sans cesser. Les Ainsi-allés des dix directions entrent en Samadhi, et les Bouddha des trois temps éprouvent du plaisir à enseigner. Avec la maîtrise des pouvoirs spirituels on fait l’expérience des Mystères. Lever les mains et bouger les pieds deviennent des Mudra secrets. Les sons provoqués par l’ouverture de la bouche sont tous des Shingons. Les pensées que l’on conçoit sont notre propre Samadhi. Les fonction merveilleuses de toutes les vertus sont le Mandala de notre propre esprit. Former un Mudra une fois surpasse la formation continue de tous les Mudras. Réciter un Shingon une fois surpasse aussi la récitation d’innombrables Shingons. Visualiser une fois permet d’être certain de transcender les Trois Temps, entrer dans un myriade de Samadhis et de cultiver de merveilleuses visualisations.
Si des Êtres entendent parler de ces Vertus mais sans Foi, vous devez savoir qu’ils tomberont invariablement dans l’Abiruka et détruiront leur propre Nature de Bouddha. Tous les Bouddha ne peuvent pas sauver, que dire des quidam.
Les phrase qui précèdent m'ont été transmises quand j'ai été initié. On m’en a transmis une autre version mais je ne l’ai pas écrite car je ne m’en souviens pas. L’Enseignement sur le Corps doté des Cinq Roues est fini.
L'enseignement sur les neufs syllabes et la nonuple vie dans la terre pure
Ensuite vient l’Enseignement sur la nonuple future vie dans la Terre Pure en deux parties : d’abord sur le sens de la phrase et ensuite sur le sens des syllabes.
Dans un premier temps on étudiera les sens de la phrase « OM AMIRITA TEIZEI KARA UN ». Ces neuf syllabes forment cinq mots. La syllabe initiale « ON » a trois significations : la premiere signification est le Trikaya, le deuxième est la prise de refuge et la troisème est une immense offrande. Ceci est expliqué dans le Shugokyô.
Ensuite vient l’explication des syllabes « AM » « RI » « TA ». On trouve ce sens dans le commentaire sur les dix Amrita.
Les deux syllabes suivantes (« TEI » et « ZEI ») ont six sens différents. D’abord comme A.
La première signification est “Grande vertus Majestueuse”, Amida a six bras de Vertu majestueuse. La deuxième est « Grande lumière majestueuse » en rapport à sa vive lumière ubiquitaire. Le troisième sens est « Grand pouvoir spirituel majestueux » en référence à son  pouvoir spirituel .  Ensuite il y’a « Grand pouvoir majestueux » car il a le pouvoir majestueux des six éléments. Par la vertu de promptement détruire ses ennemis on appelle Amida Nyôrai « Grand pouvoir majestueux ». Enfin  l’appelle aussi « Grande colère majestueuse » par sa capacité à pousser le « Grand rgissement du Lion » dés le premier Bhumi.
« HA » et « RA » on aussi six significations. D’abord « devenir un Bouddha » car Amida a atteint la bouddhéité il y’a très longtemps. Le second sens est « l’accomplissement d’actes » car il lie connaissance avec tous les nouveaux arrivants dans sa Terre-pure sans jamais s’arrêter. La troisième signification est « remplir une fonction » car Amida maîtrise les pouvoirs surnaturels. Le quatrième sens, est « oraison » car il accueille les personnes avec dix oraisons. Le cinquième sens est “pratique de la meditation » parce qu’il entre dans le Samadhi de la Sapience discriminatrice. Le sixième sens « faire des Vœux » car il à fait quatre vingt quatre Grands Vœux.
La dernière syllabe est composée de quatre sons : « A », « KA », « U » et « MA ». Comme elle détruisent les ennemis du Dharma, elle signifient « détruire». Elles signifient aussi « pouvoir de créer » car elles peuvent créer d’innombrables Vérités. Une autre acception est « crainte » car elles terrorisent les Mara des voies hérétiques. Le Unjigi traite de ce sujet en détail.
L’ « A » a déjà été traité, ces cents significations sont comme dans le Sutra (T19 :532b). Pour résumé il y’a 10 significations aux trois Vérités.
 
En résumé, il y’a dix significations concernant les trois vérités (vacuité, existence, voie médiane).
Au sujet de la Vérité de l’Existence et d’un de  ses niveaux de vérités une stance dit :« Les trois Vérités sur la coproduction conditionnée sont la Vérité de la Vacuité. Les trois Vérités sur la coproduction conditionnée sont la Vérité du Provisoire. Les trois Vérités sur la coproduction conditionnée sont la celle Voie Médiane. L’infini Esprit indivis est la vérité de la Vacuité. . L’infini Esprit indivis est la vérité de l’existence. . L’infini Esprit indivis est la vérité de la Voie médiane. Le Triple Mystère du Hôkkai est la Vérité du sans naissance. Le Triple Mystère du Hôkkai est la Vérité de l’Existence fondamentale. Le Triple Mystère du Hôkkai est la Vérité de la Voie médiane. Le Triple Mystère du Hôkkai est la Vérité du Mandala. »  
Aucun de ces sens n’est connu du Dharma littéral, autant ne pas penser au dix simultanément. Les trois première Vérités sont une merveilleuse méditation sur les mystérieuses Trois Vérités si on se base sur le point de vu exotérique. Mais ce sont les Enseignements  d’un ignorant.
Les trois Enseignements qui suivent, du point de vue étroit de l’Esotérisme abrégé,  utilisent le mot « infini » pour caractériser les « Mystérieuses Trois Vérités ». Tous les Enseignements, que cela soient ceux des Trois Véhicules ou ceux du Véhicule Unique ignorent complètement le étape de l’Esprit Indivis. Certains connaissent  six consciences, d’autre huit, d’autre encore neuf, et certains même en connaissent dix.
Dans le deuxième groupe de “Trois Vérités”, on postule trois « Vérités infinies » . Dans le troisième groupe, il est largement discuté du principe des trois vérités en relation à l’interpénétration du principe et des phénomènes.
Dans les « Trois Vérités » du troisième groupe, les Hôkkai des phénomènes et du principe  sont unis et expliqués. Ils englobent toutes les Natures. Etant donné que ce ne sont pas des Natures immuables dotées d’une « Nature propre » secrète, si l’on se réfère à l’  « Egalité originelle » le « Triple Mystère » du Hôshinbutsu n’existe ni en tant que sujet, ni en tant qu’objet. Mais pourtant, nous avons le terme « troisième Vérité ».
La quatrième Vérité concerne la réalité du Hôkkai inné, le Jissho Hôsshin de Sapience, c'est-à-dire la  merveilleuse vérité de l’Essence profondément mystérieuse qui est à la fois, caractéristique et fonction du Mandala non-duel du Grand Véhicule.
Ensuite, comme le « U » (qui signifie « la destruction de toutes les natures ») est incompréhensible on peut décliner le sens de « destruction de toutes les nature en six concepts : souffrance, Vacuité, non-éternité, non égo ; le cycle des quatre phases ; une incomparable existence en soi ; la nature-en-soi in-statique ; l’existence due à la coproduction conditionnée ;  la relativité.
Tel est le sens de « U ».
De plus comme toutes les natures sont originellement dotées de : permanence, extase, personnalité et pureté, l’Absolu est immuable et sans entrave. Elles sont sans allée ni venue, transcendent la coproduction conditionnée, baigne dans l’En-Soi, sont identiques à l’Espace et n’ont qu’une nature unique. On dit donc dans le Sutra (T.19 :505c) que le « U » désigne le Juyushin.
 Il y’a neuf types de destruction. Ce sont les neuf genres d’état mental, car ils ne sont pas conscients de l’inépuisable et innombrable nombre du Triple Mystère infini.
La dernière syllabe « MA » rend compte de l’incompréhensible de toute nature. En d’autres termes, le Soi est l’existence par soi ainsi que les deux Maîtres. Le Soi est l’Ego, les Sois sont les quidam.
Les Voies non-bouddhistes, les deux véhicules, les trois, l’Unique ainsi que le Véhicule du Kegon s’attachent tous aux Sois. Tandis qu’ils considèrent leur propre Véhicule comme l’Absolu existant pour soi et le Bouddha éveillé, pour le Shingon cet état d’esprit ne représente que le début . De plus le Mikkyo considère le royaume de la Sapience non-duelle d’où proviennent toutes les Natures comme n’étant ni le créateur ni le crée. Potentiellement présent partout il n’est détruit nulle part. Malgré tout, bien qu’il soit l’esprit indivis du Triple Mystère, il lui manque deux caractéristiques.
 Comment le Soi peut il exister ? A cause du contraste en le « Soi » et l’  « Autre » qui est distinct des caractéristiques de l’ « Autre », un « Soi » ne peut être compris. La syllabe « MA » signifie « nirmanakaya »
La troisième syllabe « MR » combine « R » et « MA », ce qui veut dire que les souillures ne peuvent être comprises ou que les pouvoir mystiques ne peuvent l’être aussi. « MR » signifie « nirmanakaya » et un changement des pouvoirs mystiques. “R” veut dire que toutes les nature fondamentalement pures transcendent la pureté et l’impureté. Cela signifie aussi  « samadhi » et plus particulièrement le Samadhi-lotus de la Sagesse Discriminatrice.
La quatrième syllabe « TA » indique que l’Ainsité de toutes les Natures ne peut être comprise. Ainsi le Chûron (T. 30: 36a) indique” La réalité du Monde et du Nirvana sont sans la moindre trace de différence”. Comme ils ne sont pas différents ils sont sans obstruction, et donc sans attachement. Comme il n’y a pas d’attachement il n’y a ni réalité ni libération.
La cinquième syllabe « TE » est composée du « E » et « TA ». Cette syllabe veut dire que la recherche de la Libération est incompréhensible. Comme le dit une stance (du Unjigi) : « L’unité de l’Identité (de toute chose) est appelé Ainsité. A cause de la diversité, l’Ainsité existe. Le principe est sans limite et la Sapience est infinie. Le nombre de grains de sables du Gange ne s’en rapproche même pas, et en broyant la totalité des Terres de Bouddha on en atteindrait pas le nombre. Malgré le nombre immense des gouttes d’eau, leur nature est la même. Bien que la rayons de lumière émis par une lampe ne soient pas un, par rapport à l’obscurité, ils ont de même substance. La Forme et l’Esprit sont indénombrables, la Vérité et la relativité n’ont pas de limite. Le Maître Esprit et les attributs de l’esprit en tant que subordonnés sont inépuisables. Ils s’interpénètrent comme les rais de lumière du Filet de Brahma. Excessivement difficiles à distinguer, chacun recelle les cinq Sapiences. Bien que nombreux ils sont indifférenciés, bien qu’indifférenciés ils sont nombreux. Ils sont une seule et même Ainsité. Ainsi ils sont appelés une Unique Ainsité. L’Unique est un et non-un, le sans nombre est un. L’Ainsité n’est pas la permanence d’Ainsité. C’est l’Unicité et la relativité. Tant que ce principe n’est pas expliqué, cet Enseignement sera considéré comme relatif et la Corne d’Abondance se tarira en épuisant le Véhicule aux Joyaux innombrables. Ce serait une grande perte.
 Les quatre corps innombrables et le vaste Triple-Mystère, sont trop immenses pour être décrit même si le Shumisen était le pinceau et la terre l’Encre. Originellement ils sont parfaits, inamovibles et inchangeables.
Au sujet de la l’incompréhensible de la recherché de la Libération, une stance dit:  « Les êtres des quatre modes de naissance et des six destinées ont d’innombrables qualités. Assis, debout, marchant ou allongés, ils forment des Mudras très secrets. Les conversations calmes ou agitées sont des Shingons. Le Sage et le Fou sont perspicaces. L’immersion dans l’Océan des naissances ou la lutte pour la Libération sont des Samadhis. J’ai déjà toutes les vertus et ne sont pas loin. Pourquoi chercher un autre endroit ?
La sixième syllabe est composée de la syllabe « E » et de la syllabe « SA » qui veut dire Vérité (satya en sanscrit et Ti en Chinois). « SA » exprime que la Vérité de toutes les natures est incompréhensible. A l’image des caractéristiques de toutes les Natures, la Vérité est libre d’erreurs ou faussetés. La Noble vérité de Dukkha ne peut être changée, même si le soleil devient froid et la lune chaude. Les samudaya sont en vérité des causes et ce sont les seules qu’il se puissent être. La fin des causes est la fin des résultats. La Voie de la Cessation est la Vraie Voie et il n’y a en pas d’autres.
Dans le Dainehankyô il est dit : « Réalisez que la Souffrance n’est pas Souffrance et que donc c’est l’Absolu. Les trois autres Nobles Vérités sont pareil. Distinguant les Quatre Nobles Vérités on y trouve d’innombrables caractéristiques ainsi qu’un Unique Vérité. Se reporter au Dainehankyô pour les explications.
Cela a trait avec les caractéristiques de la syllabe et la nature foncièrement non-née de toutes choses, ainsi que la raison pour laquelle l’absolu est sans caractéristiques, une Voie au-delà de la Parole, fondamentalement quiete, à la nature-propre égale. Sachez qu’il n’y a pas de Voir, pas de Cessation, pas de Réalisation, et il n’y a pas de Pratique. Voir, cesser, réaliser et pratique sont le Hôkkai mystérieux. C’est aussi la Vacuité, le relatif et la Voie Médiane. Il n’y a pas plus de Vérité que d’Illusion. Les caractéristiques du Samadhi n’ont pas de forme spécifique qui puissent être montrées. Donc on dit que la Réalité est incompréhensible.
La marque de la syllabe « E » est comme ci-dessus. La septième syllabe « HA » signifie que toutes les natures ne peuvent être appréhendées. En Sanscrit cela correpsond à « Hetu », la Cause. Si il y’a Réalisation de l’ « HA », alors il y’a compréhension du principe de la Coproduction conditionnée. Ceci est la caractéristique de cette syllabe.
Toute les natures apparaissant en succession, elles apparaissent à partir de causes. Sachant qu’il n’y a pas de coproduction conditionnée, l’inconstance est la base de toutes les natures. La raison en est expliquée dans le T30 :2b. On médite sur la coproduction conditionnée de toutes les natures depuis plusieurs points de vue. Comme tous sont non-nés, réalisez que toutes les natures ne sont qu’Esprit. La véritable caractéristique du Yui Shiki la connaissance parfaite de toute choses. C’est le Hôkkai de tous les Bouddha, le Hôkkai de l’essence de toutes les natures. On ne peut comprendre l’Etant comme une cause. En sachant cela on comprend que la cause directe est le Hôkkai. La cause secondaire est le Hôkkai.
L’explication de l’ « A » provient de la source et y retourne, telle est l’ultime conclusion. L’Enseignement sur la Syllabe « HA » part d’une extrémité et retourne à la source, telle est l’ultime conclusion. La syllabe « A » est fondée sur une source, est non-née et crée toute les natures. La syllabe « HA » fonde la cause de toutes choses, sans cause. Le commencement tout comme la fin recommencent , le sens interpose entre les deux devrait être parfaitement sû.
La huitième syllabe, « RA » signifies que touts les natures sont séprées de toutes les souillures, du sanscrit « RAJAS » qui veut dire imurété. Les souillures sont la motivation des sentiments corrompus. Ainsi on dit que les six sensations de l’œil etc motivent les six souillures de la forme etc. Si le principe « RA » est compris, on comprend que toutes les natures qui peuvent être vues, entendues, touchées ou sues sont toutes des caractéristiques impures. Comme une robe tachée devient impure, la poussière en tourbillon assombrit le ciel immense et cache le soleil et la lune. Telles est cette syllabe. Dans le Chunglun : Dans la recherche d’un sujet et d’un objet , il n’y a pas de sujet. S’il n’y a pas de Sujet, qui peut être l’Observateur des objet et différencier les forme externes ?
Comme il n’y arien qui puisse être  Sujet ou un Objet d’un Sujet, les quatre natures, conscience, perception, sentiment et attachement, aucun n’existe. Parce qu’il n’y a pas d’attachement la division des douze maillon de la coproduction conditionnée aussi n’existent pas. Donc quand vous voyez un forme, il s’agit d’une caractéristique du Nirvana. Les autres exemples sont similaires.
Ensuite, toutes les natures sont le pur Hôkkai de Beiraoshano, combien plus le sont les six sens impurs de l’Ainsi-allé. L’Anguli-maliya sutra précise : « Doté d’un œil permanent et eternel, le Bouddha voit clairement les formes éternelles… Son Esprit aussi est comme cela. »
C’est la veritable signification de la syllabe « RA ». La neuvième syllabe « HUM » veut dire Trikaya. Je vias expliquer cela brièvement : la syllabe « HA » est le corps de l’  « HUM », son Juyûshin. Il se subdivise en « A » qui est le Hôsshin, en « U » qui le Juyûshin et « MA » qui est le Ôjin. Le Hôsshin englobe les trois modalités et  existe en quatre formes : le Svabhavakaya, le samboghakaya, le nirmanakaya et le ninyandakaya. Ces quatre corps sont appelés le Dharmakaya. Pourquoi ? En vers:
Partout les six élements compénètrent sages et fous.
Pareillement établis, ils ne croissent pas plus qu’ils ne décroissent,
L’Esprit indivis est le svabhava-dharmakaya Buddha,
L’Essence unique est le corps de rétribution, le Samboghakaya,
La Caractéristique unique est le Nirmanakaya Bouddha qui est soumis au changement,
La Fonction unique est le Niyandakay d’égalité.
Ces quatre Corps sont inclus dans la signification de ce qui Eveille,
Les gens ordinaires des six éléments sont ceux qui doivent être Eveillés,
Les Trois Mystères qui Eveillent englobent ceux qui doivent être Eveillés,
Les quatre mandalas à réaliser compénètrent ce qui peut Eveiller.
Chacun s’interpénètre et forme un mandala.
Les trois sortes de qualités sont l’atteinte de la Bouddhéité.
Les trois Mystères indestructibles comme le Vajra, englobent le Hôkkai et ne choisissent pas un Monde avec ou sans Bouddha.
Les Gyôja des Cinq Mystères résident dans la Palais de l’Esprit.
Il n’y a pas de distinctions entre l’ornementation par  les Mystères et la non ornementation par les Mystères.
Il y’a aussi cinq sortes de Hôsshin, le Hôkkaishin étant unis aux quatre corps déjà mentionnés. Il y’a cinq sortes de Mandalas, le Hôkkaimandara étant ajouté aux quatre Mandala déjà mentionnés. Une verset du Shôikyô ( T. 18, No. 291) dit:  “Les trente-six vertus dharmiques du Jishô shin, du Juyû shin, du Ôjin sont toutes également le Jishô shin. Comme elles sont unies au Hôkkai shin, elles deviennent 37 vertus. » Le Raisan kyô (T. 18, No. 878): “Avec le Juyû Shin, le Hôkkai Shin est aussi établi ».  C’étaient les preuves scripturaires de l’existence du Hôkkai Shin en plus des quatre autres corps. Ce Hôkkai Shin est le Hôsshin des six éléments.
Venons en au Amrita Mandala du Shingon de neuf syllabes. Contemplez le toujours et gardez un Coeur sincère. Visualisez le « A » en esprit qui devient une tour faite des sept matières précieuses. Stabilisez la vision du « VAM » puis établissez le Bija « A ». Faites onduler décorations et bannières, aspergez avec la précieuse eau de purification . Des Cieux tombent alors de merveilleux vêtements et des hommes brulant de l’encens campka. Les fleurs sont des quatres couleurs rares et les oiseaux chantent sur six tons différents. Il y’a de la Joie dans la trace des nuages et sous les frondaisons. Dans le palais et ses jardins on danse. Chaque strophe de l’Enseignement du Bouddha est excellent. Des Mandarava et des manjasaka tombent, les cavernes pour la méditation sont sont calmes de l’eau de la méditation constante. Le courant précieux des huit vertus est réalisé, les « streamers » résonnent, combinant les six rythmes. L’eau du lac éxplique les six Paramita.  Les précieux vases révèlent cinq tiges et les bougies brulent les cinq sapiences. Le dais central ouvre ses huit pétales, visualisez dessus les neuf « HRIH ». Au dessus se trouve Kannon et sur les pétales sont les huit Bouddha de méditation.
Sur le cercle de huit petals suivant sont “AMRITA TEJE HARA HUM”, c’est le corps de huit divinités: Kannon Bosatsu, Miroku, Kokuzô, Fugen, Kongoshû, Monju, Sarvaranarana et Jizô. Les douze grands Bosatsu d’offrande sont arrangés dans l’ordre.
Ce Mandala est le lotus inné de tous les êtres vivants, la substance de l’Eveil et de l’ultime Extase, la grande assemblée de la pureté vaste comme l’Océan, les vingt-cinq Bosatsu, et l’assemblée vaste comme l’Océan  de la montagne ornée de fleurs (J¥j¥seikyø; Dainihonzokuzøkyømanjizokuzø, 87–4: 294, left b). Gardez cela à l’esprit jour et nuit. Le pouvoir du Shingon « BHU KHAM » change tout cela dans Terre-pure de l’Extase absolue.
Image du mandala
Cette syllabe génère les 48 vœux du étape causal de Dharmâkara (T. 20: 267c).
Les Vents de cet Enseignement émanent de la Syllabe “HA”. Avoir le « HA » est la syllabe « HRIH » d’où émane à son tour le Mandala de neuf syllabes, d’où émane à son tour le Shingon de 130 syllabes, la Grande Dharani de l’Amrita.
Question :Comment devons nous comprendre le ses de cette Dharani ?
Réponse :Le sens du texte sanscrit est le suivant :
« Namo ratna trayaya » veut dire « prendre refuge » et « s’incliner devant les Trois Joyaux », « se sauver soi-même », « de rendre hommage », « montrer du respect » etc. « Namo arya » signifie « muni », « le  Grand Sage (Shaka Nyôrai) ».  « Amitabhaya » équivaut à « lumière infinie » et aprama√åbha veut dire « disciple infini et amirta la nourriture spirituelle infinie ».  « tathagataya » signifie « le véhicule de l’Ainsi-allé ». « arhate samyaksambuddha » veut dire « destruction des ennemis », « le non-né », « approprié », « offrandes », « Eveil parfait  etc». « Tad yatha » veut dire que le Shingon peut être expliqué comme suit : « om » peut vouloir dire :   « trikaya », « atteindre l’Eveil parfait », « faire une offrande » ou « prendre refuge ». « Amrite », « continuer à vivre », « ne pas vieillir » ou « ne pas mourir et ainsi de suite ». « Amiritodbhave » peut s’expliquer par :  « trône orné de joyaux », « s’asseoir », « plaisir », « siège » « confortable ». « Amirita sambhave » veux dire :  « être né », « venir », « attirer », « exister ». « Amritagarbhe »  peut être remplacé par « Gaganaganja » ou « Ksitagarbha », « Akashagarbha » ou « Vajragarbha ». « Amrta-siddhe » signifie « accompli », « allé », « atteindre un résultat ou une cause ». « Amrta-teje » veut dire « vertus exaltées », « lumière sublime », « noble pouvoir », « force royale ». « Amrta-vikrante » se traduit par « extase sublime », « vie paisible »,  « nirvana ». « Amrita –vikranta » veut que peu importe la manière dont atteint à tous les plaisirs, c’est de toutes les façons, appelé la Joie Suprême. « Gamine » à pour sens « Espace » « demeurer », « un monde séparé de la souffrance », « reposer adéquatement dans la récitation ». « Amrta-gagana-kartikare » signifie « comme l’espace », « sans empêchement ou opposition », « obtenir la Vie ». « Amrta-duu-dubhisvare » peut vouloir dire : « son plaisant », « prêcher le merveilleux Enseignement » et « prendre plaisir à l’Enseignement ». « Sarva-arthasadha » veut dire «Tout accomplir », « accomplissement », « être rempli de la joie du Samadhi ». « Sarva-karma-kleΩa-k≈ayaμ » est traduisible par « faire que ses actes soient universels », « protéger une naissance précieuse », « nourrir la vie », « révérée assemblée des vivants », « fin des temps ». « Svaha » pour un pratiquant de pure Foi et influencé par le Bouddha, veut dire de remplir le Vœux d’accueil les autres dans la Terre-pure, d’y attirer les gens dangereux. C’était l’exposition brêve su sens de la phrase du Shingon.
.
Le Mudra Fondamental et Secret est comme suit: Les deux mains s’entrecroisent vers l’extérieur, les deux majeurs sont en forme de Lotus. Ce Mudra est appelé « la Bodaishin »,  « le Mudra de la future naissance dans la Terre-pure » ou « le Mudra générique de ceux des neufs classes dans la Terre-pure »

L'obtention de mérites incomparables (fait)
Cette partie se rapporte aux mérites que l’on obtient dès l'instant que l’on pratique la méthode des Shingon.
A cause de l’inépuisabilité (Mujin) des six Paramita et des quatre moyens (de captation) (Shishô), la Sapience du Bouddha explicite (Kenbuchie) se révèle au longs des innombrables Grands Eons (Muryo asôgi kô) et nécessite une pratique éternelle (Jôgô). De toutes les voies menant à la Sagesse, celle du Shingon seule est suprême, parce que la pratique du Triple Mystère est Véritable (Shinjitsu).
Question : 
La pratique unifiée du Triple Mystère permet d’obtenir des mérites innombrables. Mais s'il est un Gyôja qui ne fait que réciter les Shingon, ou ne fait que former les Mudra ou ne fait que s’exercer à la visualisation et qu’au surplus, il est doté de peu de discernement ? Ou alors, si quelqu’un d’intelligent ne pratique qu’imparfaitement deux des trois mystères ? Quelle est la différence de proportion dans les mérites obtenus si l’on pratique seulement un des trois Mystères ?
Réponse : 
Même avec une pratique partielle (Henshû), inattentive (Henen), si on a la Foi, les mérites obtenus dépassent ce qui peut l’être par l'Exotérisme pendant d'innombrables Éons. Cependant si l'on forme un seul doute (Issho giwaku) au sujet des Shingon, on génère un karma qui mène instantanément à la chute.
En conséquence, la clef qui ouvre l’accès par les Shingon doit être cachée dans un écrin et jetée au fond d’une source (Sen).
Du perfectionnement par les pratiques secrètes (fait)
Si le Gyôja pratiquant les Shingon manque de sagesse profonde (Jinchi) et qu'il est affligé des obstacles nés de l'ignorance et des péchés passés et présents, si il a un peu de Foi et si parfois il lui arrive de psalmodier, de former des mudra et de visualiser les formes des trois sortes de corps ésotérique (syllabes, mudra, forme corporelle) il obtient la Pureté (Shôjô).
En passant le seuil de cet Enseignement, les trois incommensurables Eons sont franchis en un instant par la commémoration de l’  « A ». En effet, le Triple-Mystère Adamantin recèle la Sapience et des Mérites sans nombre. Les quatre-vingt milles souillures (Jinrô) deviennent du beurre clarifié. Les cinq Skandhas se changent en Sapience de Bouddha (Butte). Le Shingon raisonnant de la bouche ouverte efface le Karma des crimes antiques. Le Mudra né des mains et des jambes actives apporte la joie (Zôfuku). La merveilleuse visualisation naît de l’esprit, l'Activité Mentale devient Samadhi et atteint à la perfection.
Dans le cour dégoûtante, la femme indigente (Hinnyo), tout d'un coup dresse la bannière du Hôju, et dans le vide ténébreux de l’Inscience, accroche la Lampe-des-Deux-Luminaires (Nichigatsu Tô). L’armée des quatre Maras est entravée (Menbaku). Les brigands des six objets des sens capitulent et deviennent des alliés. On peut alors espérer atteindre la contrée du Maître Esprit et la joie in-crée, et on obtient alors les quatre Hôsshin et des mérites aussi nombreux que les grains de sables du Ganges.
Au sujet des bienfaits de recevoir et chanter des Shingon, il est dit dans le Yugayugikyô (T18, No 867: 260b) qu'ils sont :
« [...]comme l'Esprit de tous les Bouddha, comme le Ôshin de tous les Bouddhas, comme cent-mille kotis de reliques de Bouddhas indescriptibles, comme les Shingon des Bouddhas, comme les actes et la pensée des Bouddhas. Tous les actes sont comme ceux des Bouddhas. Chaque paroles proférée devient un Shingon, le mouvement des membres devient la composition de merveilleux mudras. Ce qui est vu par les yeux devient le Kongôkai, ce qui est touché par le corps devient un grand mudra (mahamudra/daishuin) ». 
Ce qui précède ne vient pas des textes, mais de ma propre expérience. Je prie que ce qui font preuve de suffisamment de sagesse n'en doutent pas.
Enseignement sur l'obtention de la plus haute des neuf naissances dans la Terre-pure

Il n’y a pas d’autre Voie vers la Renaissance que celle qui consiste à compter sur les voeux fondés sur la Compassion de Dainichi et à se reposer sur ceux d’Amida.  La reine Vaidehi (Idai) et le laïc Somachattra (Gatsugai) ont obtenus la Renaissance en leur vie, tandis que Nagarjuna (Ryûju) et Dharmapala (Gohô) s'attendaient à l’obtenir après leur mort.
Question : Quels sont les Grands Vœux qui permettent au pratiquant du Nenbutsu Mystérique (Nenmitsugyô), désireux d’oeuvrer pour tous les êtres, d'accomplir la renaissance (Tôshen) dans la Terre-Pure?
Réponse : Les quatre transferts de mérites (Kaikô) sont la cause immédiate (Shin’in) de la renaissance dans la Terre-Pure.
Il y a tout d’abord la récitation des quatre Shingons incommensurables. Elle se fait en en partageant les mérites avec la totalité des êtres dans le but de devenir comme les quatre grands Boddhisattvas. Cela est le transfert des mérites avec le coeur le plus pur et la plus grande foi. 
Ensuite, alors que l'on constate le déclin du Dharma, on fait le voeu d’être pareil au Grand Maître dans la restauration et la propagation de la Loi. Cela est le transfert des mérites avec le coeur le plus pur et la plus grande foi. 
Troisièmement, il s’agit de vouloir que tous les être du Hôkkai réalisent l’Eveil sans supérieur. Cela est le transfert des mérites avec le coeur le plus pur et la plus grande foi. 
Enfin il faut, pour soi et autrui, planter suffisamment de racines de Bien pour avoir la présence d'esprit au moment de la mort de manifester la volonté de renaître au Gokuraku. Cela est le transfert des mérites avec le coeur le plus pur et la plus grande foi. 
Arrivé à la toute fin, récitez consciencieusement (Nenjô) le Gorinkuji tout en nouant les quatre Mudras. Fixez votre esprit sur le Gokuraku, arrêtez le flot des pensées et attendez la rupture du Marman (Damatsuma). 
Les quatre Mudras sont: Vajranjali (Kongôgasshô), Vajrabandha (Kongôbaku), l'Ouverture de l’Esprit (Kaishin) et l'Entrée dans la Sapience (Nyûchi). Quant aux mantras, il s’agit d’un secret.
Question : Quelle est celle des neuf classes dans la quelle renaît le Gyôja du Shingon quand il arrive au Gokuraku ?
Réponse : La plupart renaissent dans les trois plus hautes. Dans le Sammajihô, il est dit que l'on aboutit à la Terre de la Joie (Kangi ji) en ce Monde. Ce fut le cas pour Nagarjuna (Ryuju).
Question : Quels actes mènent à la naissance au Gokuraku ?
Réponse : Les trois refuges et le cinq préceptes (Sankigokai) y mènent. Les six paramitas (Rokygyô), les quatre concentrations (Shizen), les dix vertus (Jûzen), la contemplation de l'absence de Soi (Mugakan) etc. y mènent aussi. Il en va de même pour la méditation sur les quatre nobles vérités (Shitai) et sur les douze liens (Jûni innen) de la coproduction conditionnée.
Des Pratiquants dévoués au Bien d'autrui, tels que Dharmapala (Gohô) et Silabhadra (Kaigen) étaient nés dans la Terre-Pure. Des Pratiquants munis de la Bodhicitta sans origine tels que [Nagar]Juna (Juna) et [Arya]Deva (Daiba) étaient nés dans la Terre-Pure. Les Pratiquants du Véhicule Unique inconditionné (Muigyô), s’adonnant à la Contemplation sur les trois Hierogrammes (Sanjikan) (« A »,  « Mr » et « TA ») observent respectivement: la Vacuité, la Vérité provisoire et la  Voie Médiane. 
Nanyue (Nangaku) du Tiantai (Tendai) sont nés dans la Terre-Pure. Les Gyôja du Hôkkai , dénués absolument de Nature-propre (Gokumûjishô): Fa-Tsiang et Cheng Kuan du Kegon étaient nés dans la Terre-Pure.
Quand au  Himitsu Shôgon [Jôdô], les pratiquants qui ont réalisés intérieurement les Trois Mystères (Naishô sanmitsu) sont nés dans la Terre-Pure. Ainsi Jichie et Shinnen sont d'abord nés dans le Gokuraku avant de résider au Tosotsu[Ten]. 
Trop diversifier l’objet de ses études (Zôgaku) entraine la confusion (Wakushin), ce n'est pas pour autant que cette vie (Isshô) en devient infertile (Kûka). Cependant les racines de Bien ainsi cultivées, quand elles sont transférées au Gokuraku amènent à la renaissance dans la Terre-Pure de la Fainéantise (Keman Jôdô). Celui qui renonce (Fugen) à ce monde corrompu (Shaba) s’élève vers le Gokuraku. Si l'esprit d’incertitude (Giwakushin) croit sur le terreau des bonnes facultés, le transfert des mérites mène dans les marches de la Terre-Pure (Henchi Jôdô), prémisse à l’entrée au Sukhavati.
Question : Le passage du Dainichi Kyô sur les dix stades de l’esprit (Jûjushin) est congruent avec les écrits (Kyôron) de sens profond et étroit (Senshingi). Cela est il valable pour le pratiquant des mantras qui applique les enseignements graduels (Shingon shidai gyôja)  ?
Réponse : Pour dire vrai ces étapes sont originellement celles de la réalisation graduelle. Nonobstant elles s’appliquent tout à fait aux écritures de sens étroit et profond.
Le Gyôja muni de la Bodaishin non-apparue, abandonne les Huit négations (Bâbu) ainsi que l'ignorance. Il poursuit la clarté grandissante (Tenmyô) de la Voie unique(Ichidô). Il est témoin de l’illumination silencieuse inconditionnée(Mui Jakukô) et réalise l'Esprit fondamental permanent(Honji jôshin). Il réside alors éternellement sur le mont Grdhrakûta(Ryôzen) où le feu de la destruction(Kôka) ne brûle pas. Des fleurs Mandarava(Manda) et Manjaka(Manju) pleuvent jour et nuit et l’on y voit Visuddhacaritra(Jôgyô) et ses compagnons surgir de la terre(Jûchi yoshutsu): on y réalise l’illumination graduelle. L’Être qui recherche l’Eveil(Gugaku Satta) n’entretien pas longtemps de suspicions. Il n’y a pas de rapport dans ce que viens d’énoncer avec la « Doctrine particulière(Bekkyô) » et la doctrine « Sans personne pour en obtenir[le fruit](Mûgyômûjin) » du Tendai.
Pour les pratiquants du sens littéral la Réalisation de ces états est impossible, contrairement à ceux qui s’adonnent aux pratiques mystériques graduelles. La différence entre ces pratiquants revient à celle qui existe entre les bodhissattva des première et deuxièmes terres.

Ôter les obstacles démoniaques à la compréhension éveillée.
Il y a quatre sortes de démons: Les premiers sont les sujets de Mâra, les seconds sont les hérétiques, les troisièmes sont les les Démons, les quatrièmes sont les Gaki. On appelle les démons: mâras. Il créent obstacles et difficultés, et leur nom peu se traduire par "tueur(marana) du Bien". Tous ceux qui empêchent le bon travail sont des mâras. Ceux qui font de mauvaises actions sont appelés hérétiques. Ceux qui blessent le corps et empêchent de se tourner vers la Voie Bouddhique sont appelés démons. Ceux qui obscurcissent ou détruisent un esprit propre à la pratique et poussent à la superficialité sont des gaki. Y sont inclus les Bouddhas ou les Boddhisattvas des deux véhicules qui se manifestent en tant que Dieux, Humains etcaetera et ceux qui exposent le dharma exotérique et éloignent le pratiquant du Véhicule des Shingons. Ils poussent à abandonner la contemplation vigilante ainsi que les Pratiques pour de suite passer d'autres méthodes. Tout soudain on fait preuve de vigueur pour tomber aussitôt dans la lassitude. On se plait aussi au Dhyana et goûte les débats. Tels sont les actes de ces démons, mais pas tout le temps.
Les deux rouleaux du Majibon dans le Daihannygyô traitent du contrôle et de la pacification de ces démons. S'ils prennent l'avantage ils empêchent et détruisent tout ce qui est bon dans le Bouddhisme Séculaire et font chuter dans une destinée funeste.    ************PAGE 84 PAR64*****
Question: Quand les mâras frappent, comment les pacifier?
Réponse: Il y'a quatre manières en accord avec leur mode d'apparition: la pacification par l'opposition mutuelle, par contraste, par ces deux manières, par aucune des deux. Pour la première voici comment faire: Les Terres de Bouddha et de Mâra existent constamment dans les trois périodes et sont intégrales au Hôsshin Sanmitsu. Ils s'interpénètrent, on la même saveur et sont égaux. Le Sanmitsu des mâras et le mien ne sont fondamentalement pas deux. A l'origine il n'y a pas deux natures. Comment peut il y avoir d'obstruction alors que nos Trois-mystères sont fondamentalement les mêmes et sans naissance. Bouddhas et mâras sont le même Hokkai. Les Trois-mystères du Yoga s'interpénètrent l'un l'autre. Ils sont au delà de la blessure, de la destruction, de la calomnie et de la haine. Toutes les divinités irrités sont les incarnations de Dainichi Nyôrai manifestées pour la pacification des mauvaises personnes. La conduite et les actions ne sont pas sans signification symbolique, tous les sons sont de Shingons. Le sens des pensées est entièrement la sagesse du Dhyana. Toutes les caractéristiques pouvant êtres aimées, comme il en est de chaque choses, sont aussi le Corps-tournant-la-Roue de l'Ainsi-Venu. La parole proférée n'est pas sans Dharani. Se tenir debout, marcher, être assis et ête allongé sont les caractéristiques des Mudra. L'apprentissage et les pensées sont aussi Sagesse et Samadhi. Telles sont les caractéristiques de la pacification selon l'Enseignement Révélé.
Vient ensuite la pacification par l'opposition mutuelle et le contraste. Si les mains peuvent former un Mudra, les démons peuvent être évités. En chantant une Dharani ils peuvent être contrôlés. En contemplant la Sapience ils peuvent être détruits. Quelle est la méthode de contemplation? La stance dit:

" Je suis un récitant de Dharani. Le démon est une sortes de pêcheur qui brise les préceptes. Tous les Bouddhas, Boddhisattva, Sravaka, Sages, les huit classes de divinités bénéfiques, en protégeant celui garde les Enseignements me protègent pleinement. Ils ne m'abandonnent pas un instant. Le moindre avantage gagné par les démons m'expose à être empêché dans la Voie Vraie. Les démons naissent de l'Ignorance. Je pratique le Dharma depuis des temps immémoriaux. Le Gyôja est comme une lumière brillante, les démons comme l'obscurité profonde. Comme lumière et obscurité ne peuvent coexister, comme le feu consume le bois. Les activités des démons sont originellement non nées. Elles sont comme une apparition ou un rêve. Elles ressemblent à une fleur dans le ciel, les poils de la tortue. Les ténèbres ne peuvent étouffer la lumière. Comment l'erreur pourrait elle faire obstacle à la Voie Véritable? C'est la pacification par l'opposition.
La pacification par l'accord et celle par l'opposition sont utilisées en même temps. La dernière sorte de pacification est celle ne faisant appel ni à l'une ni à l'autre.

Une stance dit:  "Toutes les natures sont fondamentalement non-nées. Leurs nature propre est sans description, elles sont pures et non souillées, elles sont causées par des actions et sont comme l'Espace."

Un verset du Dainichi kyô:
 "Comme l'Esprit n'a pas de nature propre, il est au delà des causes directes et immédiates. Il est libre de l'apparition de l'Action. Son origine est celle de l'Espace." Et Aussi :"Toutes les destinées ne sont que des désignations. Il en va de même pour les signes des Bouddhas. Tous les éléments du Royaume de Bouddhas sont originellement purs. La Vérité est constante, extatique, pure et possédée d'individualité."

Mes actes comme ceux des démons sont absolument calmes et vides. Ils n’existent nulle part, sont sans attention ni activité mentale, sans attachement. Jamais ils n'augmentent ni ne décroissent, pas plus qu'ils ne sont ceci ou cela. Sans essence, forme ou fonction, ils ne sont ni libres ni entravés, pas plus qu'ils ne peuvent être détruit. Ils ne peuvent prétendre ni au Soi ni à l'altérité. Le ciel ne peut nuire au ciel. Comment la Réalité peut elle se confronter à la Réalité?
Divers Enseignements sur l'atteinte de la bouddhéité dés ce corps.
Il y a en général quatre sortes de pratique pour atteindre la bouddhéité dans ce corps. Il s'agit:
de la pratique des mudras et Shingons unis à une profonde sagesse,
de la pratique des mudras et Shingon à l'unisson de la contemplation d'un objet des sens,
de la pratique des mudra et Shingon avec une foi profonde,
de la pratique d'un mystère pour obtenir des mérites.
Pour la première, comme la Prajna est atteinte et que les Shingons et mudras sont biens exécutés, Sokushin Jobutsu est obtenu. Pour la deuxième il n'y a pas de Prajna mais comme il y'a ferveur dans la composition des mudras.
Pour la seconde,il n'y a pas de contemplation de la profonde Sagesse. Mais parce qu'avec ferveur on forme les mudras, chante les Shingons et que l'on contemple à l'unisson de la forme écrite ou du Mudra ou du Shingon on obtient Sokushinjobutsu.
Pour la troisième pratique, il n'y a rien de commun au deux précédentes.Mais par une Foi constante et profonde et coordination des Shingons et de Mudras, on devient rapidement un Bouddha.
Pour le quatrième type de pratique, il suffit de méditer à un seul avec persistance. Avec la compréhension de cet Enseignement on obtient un esprit entraîné et Sokushinjobutsu.
Même sans Sagesse ni pratique des Mudras et Shingons, avec une foi inébranlable dans le Enseignements et en contemplant la forme d'un syllabe et on devient un Bouddha. On médite sur la forme d'un Mudra et on devient un Bouddha. On médite sur les caractéristiques d'une divinité et on devient un Bouddha.
Même sans tout ce qui a été décrit plus haut, en récitant un seul Shingon ou une seul syllabe on devient un Bouddha.
Aussi en formant un seul Mudra, sans autre forme de pratique secrète, comme il y' a unité, on est certainement un Bouddha. Voilà l'explication générale.
Question : Quand on atteint l' Éveil parfait , est-ce parce que les Trois Mystères sont à l'unisson ? Sinon qu'elle en est la raison ?
Réponse : Quand l’Éveil parfait est atteint, c'est toujours dans ce corps, grâce au Trois mystères à l'unisson.
Question : Si l'on devient un Bouddha dans ce corps grâce à l' unisson des Trois Mystères, quelle en est la raison ?
Réponse : Quand on devient un Bouddha en se fondant sur une ou deux pratiques etc, ce n'est pas atteindre l' Éveil parfait. Mais pas le pouvoir mystérieux du Kaji, les pratiques manquantes sont manifestées. Les Trois Mystères sont atteints et la Bouddhéité dans ce corps est obtenue.
Question : Quels Soutra, commentaires ou traités expliquent le sens de Sokushinjobutsu?
Réponse : Le Sokushinjobutsu Gi de Kôbo Daishi.

L'Enseignement sur les capacités de ceux à qui l'on doit enseigner.
Il y’ a deux sortes personnes a convertir: Ceux nés dans la Terre-pure dans ce corps et ceux qui y renaissent immédiatement après leur mort. La première catégorie se scinde en deux : celle de ceux qui naissent dans la Terre-pure dans ce corps et ont de grandes capacités et celle ce ceux qui naissent dans la Terre-pure après la mort et ont peu de capacités. De même, chacune de ces sous-catégories est divisées en deux : les vifs et les lents d’esprit.
Ceux qui sont vifs d’esprit entrent directement dans le Samadhi de l’ « Essence du Hokkai », contemplent largement le Hokkai et atteignent Sokushinjobutsu: ils réalisent que tous les Êtres sont Eveillés de manière innée et qu’ils forment la Grande Bodaishin de Fugen Bosatsu. Leur objet de méditation est l' « A » inné et la Sapience originellement non-née est leur Essence. Sujet et Objet sont d’une seule essence et l’Esprit un seul et même royaume très pur. Réaliser le principe originellement non-né c'est couper la Voie de la Parole. Réaliser le principe originellement non-né est se libérer de tous les péchés. Réaliser le principe originellement non-né est savoir que les causes karmiques ne peuvent être appréhendées. Réaliser le principe originellement non-né est comprendre que l’Égalité avec l’Espace ne peut être appréhendée.
Quand on réalise le principe de la Vacuité, que l’on se rend compte que le Sujet et l’Objet sont de même Essence, c'est-à-dire que le sujet de la contemplation est l’Esprit indivis des Causes et que l’Objet de la contemplation est la Sphère Unique des caractéristiques de la Syllabe, On atteint alors Sokushinjobutsu. C’est le samadhi des « Grande facultés et de la large contemplation ».
Pour ceux de grande faculté mais peu de talent, ils entreront à la longue dans le samadhi de l’ « Essence du Hokkai » et contemplent les Bijas des cinq Roues innés. Ces cinq syllabes sont les quinze sortes de samadhis adamantins (T.18 :911a). Une syllabe est les Quinze, les quinze sont l’Une.
Une syllabe est les cinq. Dans cette syllabe contemplez les huit Enseignements. Une syllabe embrasse maint Enseignements, maintes syllabes s’unissent dans un Enseignement. Une syllabe explique plusieurs Enseignements, plusieurs syllabe un seul Enseignement. Une syllabe achève maints Enseignements, maintes syllabes achèvent un seul Enseignement. Une syllabe détruit maints Enseignements, maintes Syllabes détruisent un Enseignement. Par cette contemplation en succession pratiquée douze fois durant les douze séquences, chaque syllabe contemplée est consumée dans la source du Samsara. Par cette contemplation pratiquée en ordre inversé douze fois durant les douze séquences, chaque syllabe contemplée arrive à la source nu Nirvana. C’est le principe de la Non-naissance-foncière.
En entrant dans le Samadhi des « Quinze Vajra », il est possible de contempler le principe absolu fondamentalement non-né. Alors sujet comme objet s’éteignent. Comme l’ « A » est fondamentalement non-né, la Parole du « VA » est incompréhensible. Comme les expressions langagières sont impossibles, donc la pureté et l’impureté du « RA » sont incompréhensibles. Comme alors la Liberté est atteinte, les Causes karmiques du « HA » sont incompréhensibles. Comme le Non-Soi est obtenu l’égalité avec l’Espace du « KHA » est incompréhensible. Quand la tête comme la queue cessent et que l’Esprit inné est réalisé, Sokushinjobutsu est atteint. C’est la contemplation graduelle.
Les vifs d’esprit de peu de facultés révèrent différents Honzon qui représentent des attributs mentaux.
Voici ne interprétation brève en se basant sur un Kannon Bosatsu. Les neuf sortes du « HRIH » inné sont les Germes du Kannon inné en tous les Êtres. En prenant appui sur le principe qui veut que la Honte \footnote{Hrî en sancrit} est incompréhensible ils croient en un Lotus blanc pur, merveilleux et sans tache et réalisent la Fleur de lotus. Durant les quatre périodes in cultivent les Trois Mystères. Gardant le Mudra de « l’Ouverture du Lotus » et chantant le Shingon « OM VAJRA DHARMA HRIH » il méditent sur le principe qui pose que la Honte est incompréhensible et réalisent le Lotus Inné Éveillé. Ceci est appelé réaliser Sokushinjobutsu pour ceux qui ont peu de Facultés.
Les lents d’esprit avec peu de facultés entrent dans le Samadhi du Lotus et contemplent les hiérogrammes « HA, RA, Î , AK ». Durant les quatre périodes ils pratiquent ce Samadhi sans jamais oublier ces quatre syllabes majestueuses. Durant cette vie ils traversent les étapes des Seize Bosatsu ; développent l’humble et pure Bodaishin ; traversent les étapes de Kongosatta, Kongoô, Kongoragya et Kongosadhu ; contemplent la sens de la pensée d’Eveil « HA RA Î AK » et réalisent ces étapes. Ensuite ils pratiquent les quatre Pratiques Eveillées. Ils traversent les étapes de Kongoho, Kongoteja, Kongoshô et Kongohasa et réalisent les principes des syllabes « HA RA Î AK ». Ensuite il cultivent les quatre Sapiences. Il traversent les étape de Kongohô, VajratΔk≈√a, Vajrahetu, and Vajrabha≈a et atteignent les sapiences des « HA RA Î AK ». Ensuite ils cultivent les étapes de Vajrakarma, Vajrarak≈a,Vajra yak≈a, et Vajrasandhi et atteignent les quatre Actions. En seize vies ils manifestent le Lotus Inné de l’esprit. Réalisant le Samadhi du Lotus et utilisant les expédients salvifiques ils devient Dainichi Nyôrai.
On peut appliquer le raisonnement ci-dessus à d’autre divinités que Kannon Bosatsu.
Question: Les pratiquants conventionnels du Shingon et les pieux pratiquants du Nenbutsu ne sont pas forcément tous nés dans la Terre Pure. Comme utiliser son Esprit pour naître dans la Terre Pure ? Vous avez déjà expliqué que réciter une fois ou dix est la cause immédiate de la naissance dans la Terre Pure. Pour un homme ou une femme doté d’un esprit, qu’est ce qui arrête la pensée de naître dans la Terre Pure ?
Réponse : Il existe bien des causes indirectes à naissance dans la Terre Pure. Il faut utiliser son esprit habilement. Certains cultivent les pratiques du Shingon, d’autres réussissent en chantant le nom du Bouddha. Ils considèrent les personnes qui voient et entendent le Bouddha mais manquent de Foi dans la Sapience de l’Ainsi Allé. Cette pratique n’est pas une cause directe. D’autres cherchent l’admiration des autres et subissent des pratiques douloureuses pour le futur. Cela aussi n’est pas une cause directe. D’autres, encore, pour les honneurs et la forture chantent le Hokke Kyô et d’autres. Cela aussi n’est pas une cause directe. D’autres pour la gloire, respectent les préceptes, mais cela pas une cause directe. D’autres soutiennent qu’ils ont raison et ques autre ont tort. Mais cela n’est pas une cause directe. Des personnes savantes disent : « Les causes directes de dix chants sont l’intention d’une époque différente. »
Réalisez que c’est pareil à médire les Vaipulya sutras et encore, ce n’est pas une cause directe. Certains se réservent les pratiques Esotériques ou exoteriques, mais ce n’est pas une cause directe.
Les pratiquants de Amida ou Miroku sont adversaires, c’est la cause directe de la chute dans les Enfers. C’est tout comme ces Bosatsu qui discutent les deux vérités. Si l’on connait de tels usages de l’esprit qui ne pourrait pas naître dans la Terre-Pure ?
Kobô Daishi a dit : « Ignorance et Eveils existent en moi. Celui qui est libre d'appropriations est arrivé au but. »

Levée des doutes par des questions et des réponses fait
Question: 
Si l'on se réfère à l’Accès par les Cinq Roues (Gorine), combien y' a t il de sortes d'individus adaptés à sa pratique?
Réponse: 
Il y en a deux sortes. Les premiers, intelligents (Jôchi) et dotés de facultés supérieures (Jôkon) réaliseront Sokushin Jobutsu. Les seconds ont simplement la Foi et pratiquent superficiellement (Shingyô Sen) en vue de renaître dans la Terre-pure après leur mort. En prenant ces catégories comme prémisses on peu déduire bien plus de types de pratiquants. Tous renaîtrons dans le Mitsugon. Tous aspirent aux terres pures des dix directions.
Question:
Pourquoi le simple fait de dire le Shingon de Dainichi Nyôrai devient la cause immédiate (Shin in) de la renaissance dans les terres pures des dix directions?
Réponse:
Parce que  ce Shingon est la dharani des Bouddha des dix directions. Il est le Coeur des êtres des trois périodes du temps. Ainsi, si l'on répète ce Shingon à dessein, on peut renaître dans les terres pures des dix directions, comme par exemple, dans le paradis de Miroku ou les cavernes des ashura etc. Tout comme celui qui répète le Shingon d'Amida en neufs syllabes ne conçoit pas de pensée superficielle ou erratique, quand on pratique le Shingon shû, tous les mots deviennent des Mantras. Que dire alors du nom Amida? Ceux qui prononcent ce nom s'adonnent dores et déjà à toutes les pratiques. On peu dire que les trois famille de Buddha y sont contenues, ce qui entraine la connaissance de toutes les divinités.
"A, Â,AN,AKU": La première syllabe est l’Éveil qui est la Cause. Les deux syllabes suivantes sont la Grande-Compassion qui constitue le Racine. La dernière syllabe est l'expédient salvifique qui est le but ultime.
"SA, SÂ, SAN, SAKU": La première syllabe est la cause, les deux suivantes sont la racine et la dernière est le but ultime.
"BA, BÂ, BAN, BAKU": Voir ci-dessus.
Toutes les divinités, que cela soient les Bouddhas, les Bosatsu, les Kongôten encartera on chacune un Hierogramme qui ont tous quatre variations (cause, pratique, actualisation et nirvana) elles peuvent être classées en trois parties. Ces trois incommensurables catégories sont toutes la cause directe de la renaissance dans la Terre-Pure.

Question: 
Bien des Enseignements font du Triple Mystère la base de la renaissance dans la Terre-Pure. Que veut donc dire dans ce Dharma être doté du Triple Mystère?
Réponse:
Les Trois-Mystères du Hôsshin sont extrêmement profonds et subtils (Shinshinsai), 
Même les merveilleux Éveillés du Dharma Explicite n'en approchent pas.
Les six élément du Corps de Sapience (Chishin) sont vastes et mystérieux, 
Seuls la parfaite Sapience de la Doctrine Esotérique peut en rendre compte. 
Ichidô Myô Jakuko Butsu, Calme et dénué de craintes à renoncé à la parole (Dangongo),
Teishûson, originellement éveillée au regard des trois natures,
Abandonne respectueusement sa Réalisation pour rechercher L’Éveil Véritable (Shinkaku). 
Le Hoshinbutsu Nyôrai est silencieux et ne répond pas, Le Ojin Shugata garde le secret et ne dit mot. 
Les Eveillés qui occupent la demeure des futurs Bouddhas sont perplexes en songeant à de tels royaumes. 
Le porteur de la flamme de l'Enseignement (Makasyapa) est lui aussi très loin de ces endroits.
La substance de la Forme est le Mystère du Corps, les postures actives et passives sont donc les Mudras secrets. Tous les sons de la Voix sont le Mystère de la Parole, les paroles grossières ou superficielles sont donc aussi des Shingons. Les mentations pures et impures sont le Mystère de l'Esprit, donc les discriminations éveillées ou illusionnées sont Sapience. Les pensées et sentiments tus ou exprimés sont aussi le Mystère de l'Esprit, ils sont un Mandala qui comprend la totalité du Hôkkai. Les phénomènes et la talité sont fondamentalement non-duels, les contemplations propres et impropres sont donc des samadhis. La forme ne différant pas de l'Esprit ils sont sont complètement unis et s'interpénètrent à la manière de l'Espace.
Les pratiques Ésotériques ne sont pas destinées à tout le monde. Les Enseignements secrets ne doivent pas être transmis sans vergogne. La sagesse superficielle déborde et se manifeste car peu de bénédictions s'y attachent. La Sagesse inférieure, pareillement, fait des remous car elle est fautive. Pour ceux de peu d'envergure, le coffre au trésor est caché au fond du puit. Leur éloquence est bloqué dans la gorge à cause de leur manque de Foi.
L'apparition du doute est cause de renaissance aux Enfers, aussi ce n'est pas par égoïsme que je garde l'épée du proverbe. N'oubliez pas que la crainte et les vues fausse meurtrissent l'existence. Ne cachez pas le Yoga au pratiquant exotérique. Sans une Foi entière on invite les calamités. Les trésors des Trois familles ne doivent être ni négligés ni sous-estimés. Respectez et honorez le Triple Mystère. Le pouvoir de la prise de refuge est de faire entrer profondément dans l'Océan du Lotus de l'Esprit. C'est faire preuve de grande Foi que de lever les yeux mystérieusement vers la Lune Éveillée.
Fin du Gorinkujimyôhishaku
Postface
Ce texte contient des instructions données durant la cérémonie de l’Onction (Kanjo). Quelqu’un qui ne l’a pas reçu peut néanmoins prendre ses instructions auprès d’un Maître de la Loi. Le sujet de grande importance est le sens secret des cinq organes. Il doit être étudié et pratiqué avec la plus grande attention.
Après la rédaction de ce traité je suis entré en Samadhi. Quand soudain Hôshôbô Kyôjin m'apparut et il dit:
« Une fois le mon Kun Lun (Koronzan) écroulé on voit qu'il est composé de roche et de métal. Que l’on pratique le Samadhi de Dainchi Nyôrai ou celui d’Amida, Sages et sots sont tous non-duels. Je suis un vieil habitant du Ôgon Sekkai et vous être un nouveau venu au Mitsugon Jodô. En sortant de la grande forêt d'arbres Enbu aurons nous un parfum différent? »
Ayant terminé de parler,  il disparut comme s'il se fut agi d'une illusion. Sans bien comprendre pourquoi je me mis à pleurer et à ressentir une grande honte. Alors, voyant les limites du Mitsugon je réalisais la fin du samsara.
Annexes
Sur les Corps émanés
Dans son traité, Kogyô Daishi Kakuban fait abondamment allusion aux différents corps que peut revêtir Dainichi Nyôrai. Il s’agit dans ce chapitre de brosser un rapide tour d’horizon de ces différentes émanations. 
Selon la Bouddhologie classique la notion de Corps de Loi provient des temps antiques de la Bonne Loi. Shakyamuni aurait dit à ses disciples que son véritable corps était son Enseignement. Après son Parinirvana la notion de Corps de Loi s’est étoffée et a gagnée en importance. Elle a été abondamment discourue par les doctrinaires de l’école indienne du Yogacara, autrement appelée: Cittamatra (rien-que-pensée)


Théorie du Rien-que-pensée
Théorie de la Tradition de l’Ornement de splendeur Le sino-japonais « Rin » est la traduction du sanscrit « Chakra »
 Littéralement « glorifié/transfiguré par le (triple) Mystères. Le terme Mystère est employé au sens que les grecs anciens ont contribué à lui donner en tant que culte et corpus performatif secret visant à transformer son pratiquant et à lui révéler des secrets. Cependant l’acception chrétienne donne au terme une coloration bienvenue dans le sens où la pratique du Mystère donne accès à une Gnose.
 Comprendre: « par ses propres efforts », « par la pratique ascétique ». 
 Pour le composé « 此心/shishin » les dictionnaires donnent: « attitude mentale ». Cependant le dernier caractère du composé évoque le Coeur, qui était à la fois le siège de l’Esprit mais aussi des sentiments, une conflation des Hrdiya et le Citta du sanskrit. Nous avons préféré traduire la locution par « attitude spirituelle » dont le ton est riche en évocations potentielles.
 Ce terme est tiré directement de la doctrine du Kegon.
 Le texte du Taishôshinshu Daizôkyô donne « 五廣 / Gokô » qui peut prendre plusieurs sens: Cinq particularités, Cinq immensités. La dernière acception est peut-être une allusion au Cinq grands Eléments.
abbreviation de « 五相成身/ Gosô jôshinkan »: Examen mental sur la réalisation en son corps propre des qualités du Bouddha. Rituel du Kongôchô où l’on atteint les Cinq Sapiences sous la forme de cinq réalisation corporelles et mentales. 
 
 Plus littéralement accroissent la luminosité: Yûmyô.
 Le texte donne « réfutation »: Nôha.
 Ce passage fait référence au chapitre 12 du Dainehan Kyô, le Mahaparinirvana Sutra du Grand Véhicule révéré dans tout l’Extrême-orient.
 Littéralement la Lampe du Soleil et de la Lune. Est-ce une allusion au Boddhisattva du « Sutra du Lotus » ? 	
 Mère du roi Ajatasatru
 Laïc très riche qui aurait prié la triade d’Amida Nyôrai pour arrêter une épidémie sur les instances de Sakyamuni. Associé en Chine et au Japon à Yoryu Kannon. 
 Terme sanscrit désignant une partie vitale du corps.
 Il s’agit du surnom de Huisi, le deuxième patriarche de l’Ecole Tiantai. 
 Voir « Les doctrines japonaises de l’école Tendai » de J.N Robert pages 212 et 213.
 Il s’agit ici de divinités du Kongôkai mandata.
 Le Bouddha de la Calme illumination de la Voie unique
 Le Souverain à la perle (Cintamani), faut il enterre par là le Cakrarvatin?
 Il s’agit du mont Kun Lun qui au fil du temps et par l’influence du Bouddhisme est devenu l’équivalent sinitique que Mont Sumeru des indiens. Notons que pour le populaire chinois il est désigne la tête, le mental.
 La Terre-pure de Monju Bosatsu.
